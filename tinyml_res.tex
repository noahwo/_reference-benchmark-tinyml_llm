\documentclass{article}
\usepackage[margin=1in]{geometry} 
\usepackage{graphicx} % Required for inserting images
\usepackage{multirow}
\usepackage[table,xcdraw,dvipsnames]{xcolor}
\usepackage{url}
% \usetikzlibrary{positioning, arrows.meta, decorations.pathreplacing, shapes.geometric}
\usepackage{listings}
\usepackage{color}
\usepackage[section]{placeins}
\definecolor{dkgreen}{rgb}{0,0.6,0}
\definecolor{gray}{rgb}{0.5,0.5,0.5}
\definecolor{mauve}{rgb}{0.58,0,0.82}
\definecolor{airforceblue}{rgb}{0.36, 0.54, 0.66}

% Add this before the lstlisting environment
\lstdefinelanguage{JSON}{
    string=[s]{"}{"},
    stringstyle=\color{airforceblue},
    numbers=left,
    numberstyle=\small,
    basicstyle=\ttfamily\small,
    keywordstyle=\color{blue},
    keywords={false,true},
    comment=[l]{//},
    commentstyle=\color{gray},
    morecomment=[s]{/*}{*/},
}


\usepackage[breaklinks=true,colorlinks=true,linkcolor=blue]{hyperref}

\usepackage{tikz}
\usetikzlibrary{positioning, arrows.meta, decorations.pathreplacing, shapes.geometric}
\usepackage{placeins}



\title{reference-tracking-tinyml_llm}

\begin{document}
\tableofcontents
\section{Experiments Results}

\subsection{Original in Thesis}
\begin{table}[hp]
  \centering
  \resizebox{\textwidth}{!}{%
  \begin{tabular}{|l|rlll|llll|l|}
  \hline
  \multicolumn{1}{|c|}{} &
    \multicolumn{4}{c|}{Token consumption (tks)} &
    \multicolumn{4}{c|}{Time consumption (s)} &
     \\ \cline{2-9}
  \multicolumn{1}{|c|}{\multirow{2}{*}{Stages}} &
    \multicolumn{1}{r|}{Average} &
    \multicolumn{1}{l|}{Minimum} &
    \multicolumn{1}{l|}{Maximum} &
    Median &
    \multicolumn{1}{r|}{Average} &
    \multicolumn{1}{l|}{Minimum} &
    \multicolumn{1}{l|}{Maximum} &
    Median &
    \multicolumn{1}{|c|}{Success rate} \\ \hline
  Data processing &
    \multicolumn{1}{r|}{16500.56} &
    \multicolumn{1}{l|}{13072} &
    \multicolumn{1}{l|}{21032} &
    14983.50 &
    \multicolumn{1}{r|}{34.92} &
    \multicolumn{1}{l|}{24.47} &
    \multicolumn{1}{l|}{48.33} &
    33.28 &
    0.8 \\ \hline
  \begin{tabular}[c]{@{}l@{}}Model INT8 Quantization\\[-0.8ex] \& conversion\end{tabular} &
    \multicolumn{1}{r|}{5318.17} &
    \multicolumn{1}{l|}{613} &
    \multicolumn{1}{l|}{14146} &
    3597.50 &
    \multicolumn{1}{r|}{21.39} &
    \multicolumn{1}{l|}{7.27} &
    \multicolumn{1}{l|}{44.32} &
    19.00 &
    0.9 \\ \hline
  Sketch generation &
    \multicolumn{1}{r|}{31184.80} &
    \multicolumn{1}{l|}{22448} &
    \multicolumn{1}{l|}{37076} &
    33518.00 &
    \multicolumn{1}{r|}{109.21} &
    \multicolumn{1}{l|}{93.17} &
    \multicolumn{1}{l|}{143.20} &
    98.43 &
    0.25 \\ \hline
  \end{tabular}%
  }
  \caption{Token and time consumption count in different stages.}
  \label{tab:data}
  \end{table}

\subsection{Results: 2024.08.09, Model: GPT-4o}

This test was conducted through 30 runs of each task, and the statistics include both passed and failed runs.

\begin{table}[!hp]
  \centering
  \resizebox{\textwidth}{!}{%
  \begin{tabular}{|l|r|r|r|r|r|r|r|r|r|r|r|r|}
  \hline
  \multicolumn{1}{|c|}{} &
    \multicolumn{4}{c|}{Total Token Consumption} &
    \multicolumn{4}{c|}{Input Token Consumption} &
    \multicolumn{4}{c|}{Output Token Consumption} \\ \cline{2-13}
  \multicolumn{1}{|c|}{\multirow{-2}{*}{Stages}} &
    \multicolumn{1}{c|}{Ave} &
    \multicolumn{1}{c|}{Med} &
    \multicolumn{1}{c|}{Max} &
    \multicolumn{1}{c|}{Min} &
    \multicolumn{1}{c|}{Ave} &
    \multicolumn{1}{c|}{Med} &
    \multicolumn{1}{c|}{Max} &
    \multicolumn{1}{c|}{Min} &
    \multicolumn{1}{c|}{Ave} &
    \multicolumn{1}{c|}{Med} &
    \multicolumn{1}{c|}{Max} &
    \multicolumn{1}{c|}{Min} \\ \hline
  Data Processing &
    16621 & 15616 & 38380 & 817 &
    14740 & 13930 & 32402 & 612 &
    1880 & 1652 & 5978 & 205 \\ \hline
  \begin{tabular}[c]{@{}l@{}}Model INT8 Quantization\\[-0.8ex] \& conversion\end{tabular} &
    4700 & 3698 & 17020 & 582 &
    3916 & 3043 & 14927 & 333 &
    784 & 662 & 2434 & 249 \\ \hline
  Sketch Generation &
    29858 & 30542 & 46780 & 12759 &
    25675 & 26457 & 42591 & 9990 &
    4182 & 4245 & 5931 & 2496 \\ \hline
  \end{tabular}%
  }
  \caption{Token consumption.}
  \label{tab:detailed_token_consumption}
  \end{table}
\begin{table}[!h]
  \centering
  \begin{tabular}{|l|c|c|c|c|}
  \hline
  \multicolumn{1}{|c|}{} &
    \multicolumn{4}{c|}{Total Time (s)} \\ \cline{2-5}
  \multicolumn{1}{|c|}{\multirow{-2}{*}{Stages}} &
    \multicolumn{1}{c|}{Ave} &
    \multicolumn{1}{c|}{Med} &
    \multicolumn{1}{c|}{Max} &
    \multicolumn{1}{c|}{Min} \\ \hline
  Data Processing &
    39.38 & 30.39 & 121.11 & 2.81 \\ \hline
  \begin{tabular}[c]{@{}l@{}}Model INT8 Quantization\\ \& conversion\end{tabular} &
    18.62 & 14.78 & 48.33 & 6.34 \\ \hline
  Sketch Generation &
    100.99 & 108.82 & 147.56 & 64.41 \\ \hline
  \end{tabular}
  \caption{Time consumption.}
  \label{tab:detailed_token_consumption}
\end{table}

\begin{table}[!h]
\centering
\resizebox{\textwidth}{!}{%
\begin{tabular}{|l|r|r|r|r|r|r|r|r|r|r|r|r|}
\hline
\multicolumn{1}{|c|}{} &
  \multicolumn{4}{c|}{Total Cost (\$)} &
  \multicolumn{4}{c|}{Input Cost (\$)} &
  \multicolumn{4}{c|}{Output Cost (\$)} \\ \cline{2-13}
\multicolumn{1}{|c|}{\multirow{-2}{*}{Stages}} &
  \multicolumn{1}{c|}{Ave} &
  \multicolumn{1}{c|}{Med} &
  \multicolumn{1}{c|}{Max} &
  \multicolumn{1}{c|}{Min} &
  \multicolumn{1}{c|}{Ave} &
  \multicolumn{1}{c|}{Med} &
  \multicolumn{1}{c|}{Max} &
  \multicolumn{1}{c|}{Min} &
  \multicolumn{1}{c|}{Ave} &
  \multicolumn{1}{c|}{Med} &
  \multicolumn{1}{c|}{Max} &
  \multicolumn{1}{c|}{Min} \\ \hline
Data Processing &
  0.0557 & 0.0517 & 0.1408 & 0.0036 &
0.0369 & 0.0348 & 0.0810 & 0.0015 &
0.0188 & 0.0165 & 0.0598 & 0.0021 \\ \hline
\begin{tabular}[c]{@{}l@{}}\vspace{1pt}Model INT8 Quantization\\[-0.8ex] \vspace{-0.1pt}\& conversion\end{tabular} &
  0.0176 & 0.0142 & 0.0588 & 0.0033 &
  0.0098 & 0.0076 & 0.0373 & 0.0008 &
  0.0078 & 0.0066 & 0.0243 & 0.0025 \\ \hline
Sketch Generation &
  0.1078 & 0.1100 & 0.1484 & 0.0527 &
0.0653 & 0.0682 & 0.1065 & 0.0250 &
0.0424 & 0.0429 & 0.0593 & 0.0250 \\ \hline
\end{tabular}%
}
\caption{Cost in US dollars.}
\label{tab:detailed_token_consumption_08.09}
\end{table}
 
\begin{table}[!h]
  \centering
  \resizebox{\textwidth}{!}{%
  \begin{tabular}{|l|ll|ll|c|}
  \hline
  \multicolumn{1}{|c|}{} & \multicolumn{2}{l|}{Ave. Time (s)}  & \multicolumn{2}{l|}{Ave. Total Tokens}   & \multicolumn{1}{l|}{} \\ \cline{2-5}
  \multicolumn{1}{|c|}{\multirow{-2}{*}{Stages}} &
    \multicolumn{1}{l|}{Passed} &
    Failed &
    \multicolumn{1}{l|}{Passed} &
    Failed &
    \multicolumn{1}{l|}{\multirow{-2}{*}{Success Rate}} \\ \hline
  Data Processing        & \multicolumn{1}{l|}{36.00} & 69.79  & \multicolumn{1}{l|}{15284.30} & 28654.00 & 0.900                 \\ \hline
  
  \begin{tabular}[c]{@{}l@{}}Model INT8 Quantization\\ \& conversion\end{tabular} &
    \multicolumn{1}{l|}{16.63} &
    46.58 &
    \multicolumn{1}{l|}{3849.43} &
    16622.00 &
    0.933 \\ \hline
    
  Sketch Generation & \multicolumn{1}{l|}{113.86} & 119.58 & \multicolumn{1}{l|}{10094.89} & 14286.14 & 0.300 \\ \hline
  \end{tabular}%
  }
  \caption{Performance metrics regarding passed and failed runs.}
  \label{tab:success_rate}
  \end{table}
\begin{figure}[!h]
    \centering
    \includegraphics[width=0.75\linewidth]{data_analysis/2024/08.09/tex/distribution_scatter.png}
    \caption{Distribution of time and token consumption in three tasks.}
    \label{fig:scatter1}
\end{figure}

\newpage
\subsection{Results: 2024.10.02, Model: GPT-4o}
\FloatBarrier

    \begin{table}[!h]
        \centering
        \resizebox{\textwidth}{!}{%
        \begin{tabular}{|l|r|r|r|r|r|r|r|r|r|r|r|r|}
        \hline
        \multicolumn{1}{|c|}{} &
          \multicolumn{4}{c|}{Overall Token} &
          \multicolumn{4}{c|}{Input Token} &
          \multicolumn{4}{c|}{Output Token} \\ \cline{2-13}
        \multicolumn{1}{|c|}{\multirow{-2}{*}{Stages}} &
          \multicolumn{1}{c|}{Ave} &
          \multicolumn{1}{c|}{Med} &
          \multicolumn{1}{c|}{Max} &
          \multicolumn{1}{c|}{Min} &
          \multicolumn{1}{c|}{Ave} &
          \multicolumn{1}{c|}{Med} &
          \multicolumn{1}{c|}{Max} &
          \multicolumn{1}{c|}{Min} &
          \multicolumn{1}{c|}{Ave} &
          \multicolumn{1}{c|}{Med} &
          \multicolumn{1}{c|}{Max} &
          \multicolumn{1}{c|}{Min} \\ \hline
    DP & 16621 & 15616 & 38380 & 817 & 14740 & 13930 & 32402 & 612 & 1880 & 1652 & 5978 & 205 \\ \hline
MC & 4700 & 3698 & 17020 & 582 & 3916 & 3043 & 14927 & 333 & 784 & 662 & 2434 & 249 \\ \hline
SG & 13028 & 13747 & 16602 & 1731 & 9382 & 9829 & 12652 & 1255 & 3646 & 3932 & 4874 & 476 \\ \hline

        \end{tabular}%
        }
        \caption{Token consumption details regarding Overall, Input, and Output in three stages. 10.02}
        \label{tab:token_details_10.02}
        \end{table}
    

    \begin{table}[!h]
        \centering
        \resizebox{\textwidth}{!}{%
        \begin{tabular}{|l|r|r|r|r|r|r|r|}
        \hline
        \multicolumn{1}{|c|}{} &
          \multicolumn{3}{c|}{Execution Time (s)} &
          \multicolumn{3}{c|}{Token Consumption} &
          \multicolumn{1}{c|}{Success} \\ \cline{2-8}
        \multicolumn{1}{|c|}{\multirow{-2}{*}{Stages}} &
          \multicolumn{1}{c|}{Ave} &
          \multicolumn{1}{c|}{Max} &
          \multicolumn{1}{c|}{Min} &
          \multicolumn{1}{c|}{Ave} &
          \multicolumn{1}{c|}{Max} &
          \multicolumn{1}{c|}{Min} &
          \multicolumn{1}{c|}{Rate (\%)} \\ \hline
    DP & 39.38 & 121.11 & 2.81 & 16621 & 38380 & 817 & 0.900 \\ \hline
MC & 18.62 & 48.33 & 6.34 & 4700 & 17020 & 582 & 0.933 \\ \hline
SG & 117.86 & 242.87 & 7.13 & 13028 & 16602 & 1731 & 0.300 \\ \hline

        \end{tabular}%
        }
        \caption{Performance metrics including execution time, token consumption, and success rate in three stages. 10.02}
        \label{tab:performance_metrics_10.02}
        \end{table}
    

    \begin{table}[!h]
        \centering
        \resizebox{\textwidth}{!}{%
        \begin{tabular}{|l|r|r|r|r|r|r|r|r|r|r|r|r|}
        \hline
        \multicolumn{1}{|c|}{} &
          \multicolumn{4}{c|}{Overall Cost (USD cents)} &
          \multicolumn{4}{c|}{Input Cost (USD cents)} &
          \multicolumn{4}{c|}{Output Cost (USD cents)} \\ \cline{2-13}
        \multicolumn{1}{|c|}{\multirow{-2}{*}{Stages}} &
          \multicolumn{1}{c|}{Ave} &
          \multicolumn{1}{c|}{Med} &
          \multicolumn{1}{c|}{Max} &
          \multicolumn{1}{c|}{Min} &
          \multicolumn{1}{c|}{Ave} &
          \multicolumn{1}{c|}{Med} &
          \multicolumn{1}{c|}{Max} &
          \multicolumn{1}{c|}{Min} &
          \multicolumn{1}{c|}{Ave} &
          \multicolumn{1}{c|}{Med} &
          \multicolumn{1}{c|}{Max} &
          \multicolumn{1}{c|}{Min} \\ \hline
    DP & 5.57 & 5.17 & 14.08 & 0.36 & 3.69 & 3.48 & 8.10 & 0.15 & 1.88 & 1.65 & 5.98 & 0.21 \\ \hline
MC & 29.39 & 23.67 & 98.03 & 5.54 & 16.32 & 12.68 & 62.20 & 1.39 & 13.07 & 11.03 & 40.57 & 4.15 \\ \hline
SG & 10.16 & 10.84 & 12.58 & 1.34 & 4.69 & 4.91 & 6.33 & 0.63 & 5.47 & 5.90 & 7.31 & 0.71 \\ \hline

        \end{tabular}%
        }
        \caption{Cost details regarding Overall, Input, and Output in three stages. 10.02}
        \label{tab:cost_details_10.02}
        \end{table}
    

    \begin{table}[!h]
        \centering
        \resizebox{\textwidth}{!}{%
        \begin{tabular}{|l|r|r|r|r|r|r|r|}
        \hline
        \multicolumn{1}{|c|}{} &
          \multicolumn{3}{c|}{Ave Execution Time (s)} &
          \multicolumn{3}{c|}{Ave Total Tokens} &
          \multicolumn{1}{c|}{Success} \\ \cline{2-8}
        \multicolumn{1}{|c|}{\multirow{-2}{*}{Stages}} &
          \multicolumn{1}{c|}{Overall} &
          \multicolumn{1}{c|}{Pass} &
          \multicolumn{1}{c|}{Fail} &
          \multicolumn{1}{c|}{Overall} &
          \multicolumn{1}{c|}{Pass} &
          \multicolumn{1}{c|}{Fail} &
          \multicolumn{1}{c|}{Rate (\%)} \\ \hline
    DP & 39.38 & 36.00 & 69.79 & 16621 & 15284 & 28654 & 0.900 \\ \hline
MC & 18.62 & 16.63 & 46.58 & 4700 & 3849 & 16622 & 0.933 \\ \hline
SG & 117.86 & 113.86 & 119.58 & 13028 & 10094 & 14286 & 0.300 \\ \hline

        \end{tabular}%
        }
        \caption{Performance metrics of passed and failed runs,including execution time, token consumption, and success rate in three stages. 10.02}
        \label{tab:performance_metrics_ave_10.02}
        \end{table}
    

    \begin{table}[!h]
        \centering
        \begin{tabular}{|l|r|r|r|r|}
        \hline
        \multicolumn{1}{|c|}{Stages} &
          \multicolumn{1}{c|}{Ave} &
          \multicolumn{1}{c|}{Med} &
          \multicolumn{1}{c|}{Max} &
          \multicolumn{1}{c|}{Min} \\ \hline
    DP & 39.38 & 30.39 & 121.11 & 2.81 \\ \hline
MC & 18.62 & 14.78 & 48.33 & 6.34 \\ \hline
SG & 117.86 & 115.15 & 242.87 & 7.13 \\ \hline

        \end{tabular}
        \caption{Time consumption (seconds) in three stages. 10.02}
        \label{tab:time_10.02}
        \end{table}
    
\begin{figure}[!h]
    \centering
    \includegraphics[width=0.75\linewidth]{data_analysis/2024/10.02/tex/scatter_plot_10.02.pdf}
    \caption{Distribution of time and token consumption in three tasks 10.02.}
    \label{fig:scatter_10.02}
\end{figure}

\begin{figure}[!h]
    \centering
    \includegraphics[width=0.75\linewidth]{data_analysis/2024/10.02/tex/bar_plot_10.02.pdf}
    \caption{Average Time and Token Consumption by Task Status. \\ \textit{The average time (top) and token consumption (bottom) are compared across three stages, broken down by overall performance (considering all runs), performance of only successful runs, and of only the failed runs. 10.02.}}
    \label{fig:bar_10.02}
\end{figure}

\FloatBarrier


\newpage
\subsection{Results: 2024.10.25, Model: GPT-4o}
\FloatBarrier

    \begin{table}[!h]
        \centering
        \resizebox{\textwidth}{!}{%
        \begin{tabular}{|l|r|r|r|r|r|r|r|r|r|r|r|r|}
        \hline
        \multicolumn{1}{|c|}{} &
          \multicolumn{4}{c|}{Overall Token} &
          \multicolumn{4}{c|}{Input Token} &
          \multicolumn{4}{c|}{Output Token} \\ \cline{2-13}
        \multicolumn{1}{|c|}{\multirow{-2}{*}{Stages}} &
          \multicolumn{1}{c|}{Ave} &
          \multicolumn{1}{c|}{Med} &
          \multicolumn{1}{c|}{Max} &
          \multicolumn{1}{c|}{Min} &
          \multicolumn{1}{c|}{Ave} &
          \multicolumn{1}{c|}{Med} &
          \multicolumn{1}{c|}{Max} &
          \multicolumn{1}{c|}{Min} &
          \multicolumn{1}{c|}{Ave} &
          \multicolumn{1}{c|}{Med} &
          \multicolumn{1}{c|}{Max} &
          \multicolumn{1}{c|}{Min} \\ \hline
    DP & 11074 & 10446 & 21777 & 8587 & 8515 & 8076 & 17517 & 6688 & 2559 & 2369 & 4447 & 1831 \\ \hline
MC & 793 & 567 & 4035 & 534 & 508 & 299 & 3463 & 299 & 284 & 268 & 572 & 235 \\ \hline
SG & 13028 & 13747 & 16602 & 1731 & 9382 & 9829 & 12652 & 1255 & 3646 & 3932 & 4874 & 476 \\ \hline

        \end{tabular}%
        }
        \caption{Token consumption details regarding Overall, Input, and Output in three stages. 10.25}
        \label{tab:token_details_10.25}
        \end{table}
    

    \begin{table}[!h]
        \centering
        \resizebox{\textwidth}{!}{%
        \begin{tabular}{|l|r|r|r|r|r|r|r|}
        \hline
        \multicolumn{1}{|c|}{} &
          \multicolumn{3}{c|}{Execution Time (s)} &
          \multicolumn{3}{c|}{Token Consumption} &
          \multicolumn{1}{c|}{Success} \\ \cline{2-8}
        \multicolumn{1}{|c|}{\multirow{-2}{*}{Stages}} &
          \multicolumn{1}{c|}{Ave} &
          \multicolumn{1}{c|}{Max} &
          \multicolumn{1}{c|}{Min} &
          \multicolumn{1}{c|}{Ave} &
          \multicolumn{1}{c|}{Max} &
          \multicolumn{1}{c|}{Min} &
          \multicolumn{1}{c|}{Rate (\%)} \\ \hline
    DP & 44.17 & 73.88 & 31.68 & 11074 & 21777 & 8587 & 1.000 \\ \hline
MC & 8.87 & 22.08 & 6.66 & 793 & 4035 & 534 & 1.000 \\ \hline
SG & 117.86 & 242.87 & 7.13 & 13028 & 16602 & 1731 & 0.300 \\ \hline

        \end{tabular}%
        }
        \caption{Performance metrics including execution time, token consumption, and success rate in three stages. 10.25}
        \label{tab:performance_metrics_10.25}
        \end{table}
    

    \begin{table}[!h]
        \centering
        \resizebox{\textwidth}{!}{%
        \begin{tabular}{|l|r|r|r|r|r|r|r|r|r|r|r|r|}
        \hline
        \multicolumn{1}{|c|}{} &
          \multicolumn{4}{c|}{Overall Cost (USD cents)} &
          \multicolumn{4}{c|}{Input Cost (USD cents)} &
          \multicolumn{4}{c|}{Output Cost (USD cents)} \\ \cline{2-13}
        \multicolumn{1}{|c|}{\multirow{-2}{*}{Stages}} &
          \multicolumn{1}{c|}{Ave} &
          \multicolumn{1}{c|}{Med} &
          \multicolumn{1}{c|}{Max} &
          \multicolumn{1}{c|}{Min} &
          \multicolumn{1}{c|}{Ave} &
          \multicolumn{1}{c|}{Med} &
          \multicolumn{1}{c|}{Max} &
          \multicolumn{1}{c|}{Min} &
          \multicolumn{1}{c|}{Ave} &
          \multicolumn{1}{c|}{Med} &
          \multicolumn{1}{c|}{Max} &
          \multicolumn{1}{c|}{Min} \\ \hline
    DP & 8.10 & 7.54 & 15.15 & 6.12 & 4.26 & 4.04 & 8.76 & 3.34 & 3.84 & 3.55 & 6.67 & 2.75 \\ \hline
MC & 0.68 & 0.55 & 2.59 & 0.50 & 0.25 & 0.15 & 1.73 & 0.15 & 0.43 & 0.40 & 0.86 & 0.35 \\ \hline
SG & 10.16 & 10.84 & 12.58 & 1.34 & 4.69 & 4.91 & 6.33 & 0.63 & 5.47 & 5.90 & 7.31 & 0.71 \\ \hline

        \end{tabular}%
        }
        \caption{Cost details regarding Overall, Input, and Output in three stages. 10.25}
        \label{tab:cost_details_10.25}
        \end{table}
    

    \begin{table}[!h]
        \centering
        \resizebox{\textwidth}{!}{%
        \begin{tabular}{|l|r|r|r|r|r|r|r|}
        \hline
        \multicolumn{1}{|c|}{} &
          \multicolumn{3}{c|}{Ave Execution Time (s)} &
          \multicolumn{3}{c|}{Ave Total Tokens} &
          \multicolumn{1}{c|}{Success} \\ \cline{2-8}
        \multicolumn{1}{|c|}{\multirow{-2}{*}{Stages}} &
          \multicolumn{1}{c|}{Overall} &
          \multicolumn{1}{c|}{Pass} &
          \multicolumn{1}{c|}{Fail} &
          \multicolumn{1}{c|}{Overall} &
          \multicolumn{1}{c|}{Pass} &
          \multicolumn{1}{c|}{Fail} &
          \multicolumn{1}{c|}{Rate (\%)} \\ \hline
    DP & 44.17 & 44.17 & 0.00 & 11074 & 11074 & 0 & 1.000 \\ \hline
MC & 8.87 & 8.87 & 0.00 & 793 & 793 & 0 & 1.000 \\ \hline
SG & 117.86 & 113.86 & 119.58 & 13028 & 10094 & 14286 & 0.300 \\ \hline

        \end{tabular}%
        }
        \caption{Performance metrics of passed and failed runs,including execution time, token consumption, and success rate in three stages. 10.25}
        \label{tab:performance_metrics_ave_10.25}
        \end{table}
    

    \begin{table}[!h]
        \centering
        \begin{tabular}{|l|r|r|r|r|}
        \hline
        \multicolumn{1}{|c|}{Stages} &
          \multicolumn{1}{c|}{Ave} &
          \multicolumn{1}{c|}{Med} &
          \multicolumn{1}{c|}{Max} &
          \multicolumn{1}{c|}{Min} \\ \hline
    DP & 44.17 & 40.21 & 73.88 & 31.68 \\ \hline
MC & 8.87 & 7.98 & 22.08 & 6.66 \\ \hline
SG & 117.86 & 115.15 & 242.87 & 7.13 \\ \hline

        \end{tabular}
        \caption{Time consumption (seconds) in three stages. 10.25}
        \label{tab:time_10.25}
        \end{table}
    

\begin{figure}[!h]
    \centering
    \includegraphics[width=0.75\linewidth]{data_analysis/2024/10.25/tex/scatter_plot_10.25.pdf}
    \caption{Distribution of time and token consumption in three tasks 10.25.}
    \label{fig:scatter_10.25}
\end{figure}

\begin{figure}[!h]
    \centering
    \includegraphics[width=0.75\linewidth]{data_analysis/2024/10.25/tex/bar_plot_10.25.pdf}
    \caption{Average Time and Token Consumption by Task Status. \\ \textit{The average time (top) and token consumption (bottom) are compared across three stages, broken down by overall performance (considering all runs), performance of only successful runs, and of only the failed runs. 10.25.}}
    \label{fig:bar_10.25}
\end{figure}

\FloatBarrier


\newpage
\subsection{Results: 2024.10.28, Model: GPT-4o}
\FloatBarrier

    \begin{table}[!h]
        \centering
        \resizebox{\textwidth}{!}{%
        \begin{tabular}{|l|r|r|r|r|r|r|r|r|r|r|r|r|}
        \hline
        \multicolumn{1}{|c|}{} &
          \multicolumn{4}{c|}{Overall Token} &
          \multicolumn{4}{c|}{Input Token} &
          \multicolumn{4}{c|}{Output Token} \\ \cline{2-13}
        \multicolumn{1}{|c|}{\multirow{-2}{*}{Stages}} &
          \multicolumn{1}{c|}{Ave} &
          \multicolumn{1}{c|}{Med} &
          \multicolumn{1}{c|}{Max} &
          \multicolumn{1}{c|}{Min} &
          \multicolumn{1}{c|}{Ave} &
          \multicolumn{1}{c|}{Med} &
          \multicolumn{1}{c|}{Max} &
          \multicolumn{1}{c|}{Min} &
          \multicolumn{1}{c|}{Ave} &
          \multicolumn{1}{c|}{Med} &
          \multicolumn{1}{c|}{Max} &
          \multicolumn{1}{c|}{Min} \\ \hline
    DP & 10832 & 9433 & 25086 & 8560 & 8273 & 7030 & 18952 & 6562 & 2558 & 2319 & 6134 & 1953 \\ \hline
MC & 688 & 573 & 3949 & 545 & 403 & 299 & 3433 & 299 & 285 & 274 & 516 & 246 \\ \hline
SG & 13320 & 13906 & 17181 & 1840 & 9606 & 10011 & 13021 & 1361 & 3714 & 3916 & 4645 & 479 \\ \hline

        \end{tabular}%
        }
        \caption{Token consumption details regarding Overall, Input, and Output in three stages. 10.28}
        \label{tab:token_details_10.28}
        \end{table}
    

    \begin{table}[!h]
        \centering
        \resizebox{\textwidth}{!}{%
        \begin{tabular}{|l|r|r|r|r|r|r|r|}
        \hline
        \multicolumn{1}{|c|}{} &
          \multicolumn{3}{c|}{Execution Time (s)} &
          \multicolumn{3}{c|}{Token Consumption} &
          \multicolumn{1}{c|}{Success} \\ \cline{2-8}
        \multicolumn{1}{|c|}{\multirow{-2}{*}{Stages}} &
          \multicolumn{1}{c|}{Ave} &
          \multicolumn{1}{c|}{Max} &
          \multicolumn{1}{c|}{Min} &
          \multicolumn{1}{c|}{Ave} &
          \multicolumn{1}{c|}{Max} &
          \multicolumn{1}{c|}{Min} &
          \multicolumn{1}{c|}{Rate (\%)} \\ \hline
    DP & 47.76 & 155.93 & 32.58 & 10832 & 25086 & 8560 & 0.900 \\ \hline
MC & 6.09 & 10.21 & 3.65 & 688 & 3949 & 545 & 1.000 \\ \hline
SG & 60.55 & 87.92 & 7.73 & 13320 & 17181 & 1840 & 0.367 \\ \hline

        \end{tabular}%
        }
        \caption{Performance metrics including execution time, token consumption, and success rate in three stages. 10.28}
        \label{tab:performance_metrics_10.28}
        \end{table}
    

    \begin{table}[!h]
        \centering
        \resizebox{\textwidth}{!}{%
        \begin{tabular}{|l|r|r|r|r|r|r|r|r|r|r|r|r|}
        \hline
        \multicolumn{1}{|c|}{} &
          \multicolumn{4}{c|}{Overall Cost (USD cents)} &
          \multicolumn{4}{c|}{Input Cost (USD cents)} &
          \multicolumn{4}{c|}{Output Cost (USD cents)} \\ \cline{2-13}
        \multicolumn{1}{|c|}{\multirow{-2}{*}{Stages}} &
          \multicolumn{1}{c|}{Ave} &
          \multicolumn{1}{c|}{Med} &
          \multicolumn{1}{c|}{Max} &
          \multicolumn{1}{c|}{Min} &
          \multicolumn{1}{c|}{Ave} &
          \multicolumn{1}{c|}{Med} &
          \multicolumn{1}{c|}{Max} &
          \multicolumn{1}{c|}{Min} &
          \multicolumn{1}{c|}{Ave} &
          \multicolumn{1}{c|}{Med} &
          \multicolumn{1}{c|}{Max} &
          \multicolumn{1}{c|}{Min} \\ \hline
    DP & 7.97 & 7.17 & 18.68 & 6.28 & 4.14 & 3.52 & 9.48 & 3.28 & 3.84 & 3.48 & 9.20 & 2.93 \\ \hline
MC & 0.63 & 0.56 & 2.49 & 0.52 & 0.20 & 0.15 & 1.72 & 0.15 & 0.43 & 0.41 & 0.77 & 0.37 \\ \hline
SG & 10.38 & 11.03 & 12.75 & 1.40 & 4.80 & 5.01 & 6.51 & 0.68 & 5.57 & 5.87 & 6.97 & 0.72 \\ \hline

        \end{tabular}%
        }
        \caption{Cost details regarding Overall, Input, and Output in three stages. 10.28}
        \label{tab:cost_details_10.28}
        \end{table}
    

    \begin{table}[!h]
        \centering
        \resizebox{\textwidth}{!}{%
        \begin{tabular}{|l|r|r|r|r|r|r|r|}
        \hline
        \multicolumn{1}{|c|}{} &
          \multicolumn{3}{c|}{Ave Execution Time (s)} &
          \multicolumn{3}{c|}{Ave Total Tokens} &
          \multicolumn{1}{c|}{Success} \\ \cline{2-8}
        \multicolumn{1}{|c|}{\multirow{-2}{*}{Stages}} &
          \multicolumn{1}{c|}{Overall} &
          \multicolumn{1}{c|}{Pass} &
          \multicolumn{1}{c|}{Fail} &
          \multicolumn{1}{c|}{Overall} &
          \multicolumn{1}{c|}{Pass} &
          \multicolumn{1}{c|}{Fail} &
          \multicolumn{1}{c|}{Rate (\%)} \\ \hline
    DP & 47.76 & 44.75 & 74.82 & 10832 & 9731 & 20744 & 0.900 \\ \hline
MC & 6.09 & 6.09 & 0.00 & 688 & 688 & 0 & 1.000 \\ \hline
SG & 60.55 & 48.75 & 67.38 & 13320 & 11131 & 14588 & 0.367 \\ \hline

        \end{tabular}%
        }
        \caption{Performance metrics of passed and failed runs,including execution time, token consumption, and success rate in three stages. 10.28}
        \label{tab:performance_metrics_ave_10.28}
        \end{table}
    

    \begin{table}[!h]
        \centering
        \begin{tabular}{|l|r|r|r|r|}
        \hline
        \multicolumn{1}{|c|}{Stages} &
          \multicolumn{1}{c|}{Ave} &
          \multicolumn{1}{c|}{Med} &
          \multicolumn{1}{c|}{Max} &
          \multicolumn{1}{c|}{Min} \\ \hline
    DP & 47.76 & 39.69 & 155.93 & 32.58 \\ \hline
MC & 6.09 & 5.92 & 10.21 & 3.65 \\ \hline
SG & 60.55 & 64.74 & 87.92 & 7.73 \\ \hline

        \end{tabular}
        \caption{Time consumption (seconds) in three stages. 10.28}
        \label{tab:time_10.28}
        \end{table}
    

\begin{figure}[!h]
    \centering
    \includegraphics[width=0.75\linewidth]{data_analysis/2024/10.28/tex/scatter_plot_10.28.pdf}
    \caption{Distribution of time and token consumption in three tasks 10.28.}
    \label{fig:scatter_10.28}
\end{figure}

\begin{figure}[!h]
    \centering
    \includegraphics[width=0.75\linewidth]{data_analysis/2024/10.28/tex/bar_plot_10.28.pdf}
    \caption{Average Time and Token Consumption by Task Status. \\ \textit{The average time (top) and token consumption (bottom) are compared across three stages, broken down by overall performance (considering all runs), performance of only successful runs, and of only the failed runs. 10.28.}}
    \label{fig:bar_10.28}
\end{figure}

 
\FloatBarrier
\newpage
\subsection{2024.11.14, GPT-4o: Comparison of the impact of JSON structure on SG}
\textbf{\textcolor{red}{NOTICE:} The results were based on the task of sketch generation with 3 levels of JSON structure usage. From 0 to 2, higher number indicates fewer JSON structures. }

\begin{itemize}

\item[1] Sketch Generation0 or SG0 is the original prompt structure, with the \texttt{guideline} embedded in the \texttt{application\_specification}, which again embedded in the task description (\texttt{task}), under \#\#\# OBJECTIVE \#\#\#. They three are all JSON-like objects.  

\begin{lstlisting}[language=JSON]
### OBJECTIVE ###
"task":{
    ...
    "application_specification":{
        ...
        "guideline":{
            ...
        }
    }
}
\end{lstlisting}

\item[2] Sketch Generation1 or SG1 is the prompt structure with the task description directly written as a prompt section (noted as `\#\#\# OBJECTIVE \#\#\#`) in Markdown format, and the \texttt{application\_specification} written as another prompt section (noted as \texttt{\#\#\# APPLICATION SPECIFICATIONS \#\#\#}) but still as a JSON-like object.  The \texttt{guideline}, as a JSON-like object, embedded in the \texttt{application\_specification}, remained the same.

\begin{lstlisting}[language=JSON]
### OBJECTIVE ###
- **TASK**:...
...

### APPLICATION SPECIFICATIONS ###
"application_specification":{
        ...
        "guideline":{
            ...
        }
    }
\end{lstlisting}

\item[3] Sketch Generation2 or SG2 is the prompt structure similar to SG1, only the \texttt{guideline} was moved out as a prompt section (noted as \texttt{\#\#\# GUIDELINE FOR SKETCH \#\#\# }) in Markdown format. 
\begin{lstlisting}[language=JSON]
### OBJECTIVE ###
- **TASK**:...
...

### APPLICATION SPECIFICATIONS ###
"application_specification":{
        ...
}

### GUIDELINE FOR SKETCH ###
# Steps...
...
\end{lstlisting}

 \textcolor{red}{\textbf{As a result}}:

 Average Token Comsumption wise: 
 \hfill $SG0 \gg SG2 > SG1$

 Success Rate wise:
 \hfill $SG1 > SG0 \gg SG2$

 Average Time Consumption wise: 
 \hfill $SG2 > SG1 > SG0$


\end{itemize}
\FloatBarrier

    \begin{table}[!h]
        \centering
        \resizebox{\textwidth}{!}{%
        \begin{tabular}{|l|r|r|r|r|r|r|r|r|r|r|r|r|}
        \hline
        \multicolumn{1}{|c|}{} &
          \multicolumn{4}{c|}{Overall Token} &
          \multicolumn{4}{c|}{Input Token} &
          \multicolumn{4}{c|}{Output Token} \\ \cline{2-13}
        \multicolumn{1}{|c|}{\multirow{-2}{*}{Stages}} &
          \multicolumn{1}{c|}{Ave} &
          \multicolumn{1}{c|}{Med} &
          \multicolumn{1}{c|}{Max} &
          \multicolumn{1}{c|}{Min} &
          \multicolumn{1}{c|}{Ave} &
          \multicolumn{1}{c|}{Med} &
          \multicolumn{1}{c|}{Max} &
          \multicolumn{1}{c|}{Min} &
          \multicolumn{1}{c|}{Ave} &
          \multicolumn{1}{c|}{Med} &
          \multicolumn{1}{c|}{Max} &
          \multicolumn{1}{c|}{Min} \\ \hline
    SG0 & 14435 & 15709 & 16953 & 9991 & 10262 & 10897 & 12641 & 6928 & 4172 & 4192 & 5095 & 2914 \\ \hline
SG1 & 18014 & 18298 & 23105 & 4739 & 14243 & 14344 & 18631 & 4181 & 3770 & 4039 & 4497 & 558 \\ \hline
SG2 & 18671 & 19890 & 26742 & 7852 & 14707 & 15515 & 22237 & 5988 & 3964 & 4157 & 4657 & 1864 \\ \hline

        \end{tabular}%
        }
        \caption{Token consumption details regarding Overall, Input, and Output in three levels of JSON structure usage. 11.14}
        \label{tab:token_details_11.14}
        \end{table}
    

    \begin{table}[!h]
        \centering
        \resizebox{\textwidth}{!}{%
        \begin{tabular}{|l|r|r|r|r|r|r|r|}
        \hline
        \multicolumn{1}{|c|}{} &
          \multicolumn{3}{c|}{Execution Time (s)} &
          \multicolumn{3}{c|}{Token Consumption} &
          \multicolumn{1}{c|}{Success} \\ \cline{2-8}
        \multicolumn{1}{|c|}{\multirow{-2}{*}{Stages}} &
          \multicolumn{1}{c|}{Ave} &
          \multicolumn{1}{c|}{Max} &
          \multicolumn{1}{c|}{Min} &
          \multicolumn{1}{c|}{Ave} &
          \multicolumn{1}{c|}{Max} &
          \multicolumn{1}{c|}{Min} &
          \multicolumn{1}{c|}{Rate (\%)} \\ \hline
    SG0 & 87.42 & 174.55 & 44.50 & 14435 & 16953 & 9991 & 0.250 \\ \hline
SG1 & 42.76 & 76.44 & 7.03 & 18014 & 23105 & 4739 & 0.300 \\ \hline
SG2 & 53.39 & 145.49 & 22.41 & 18671 & 26742 & 7852 & 0.150 \\ \hline

        \end{tabular}%
        }
        \caption{Performance metrics including execution time, token consumption, and success rate in three levels of JSON structure usage. 11.14}
        \label{tab:performance_metrics_11.14}
        \end{table}
    

    \begin{table}[!h]
        \centering
        \resizebox{\textwidth}{!}{%
        \begin{tabular}{|l|r|r|r|r|r|r|r|r|r|r|r|r|}
        \hline
        \multicolumn{1}{|c|}{} &
          \multicolumn{4}{c|}{Overall Cost (USD cents)} &
          \multicolumn{4}{c|}{Input Cost (USD cents)} &
          \multicolumn{4}{c|}{Output Cost (USD cents)} \\ \cline{2-13}
        \multicolumn{1}{|c|}{\multirow{-2}{*}{Stages}} &
          \multicolumn{1}{c|}{Ave} &
          \multicolumn{1}{c|}{Med} &
          \multicolumn{1}{c|}{Max} &
          \multicolumn{1}{c|}{Min} &
          \multicolumn{1}{c|}{Ave} &
          \multicolumn{1}{c|}{Med} &
          \multicolumn{1}{c|}{Max} &
          \multicolumn{1}{c|}{Min} &
          \multicolumn{1}{c|}{Ave} &
          \multicolumn{1}{c|}{Med} &
          \multicolumn{1}{c|}{Max} &
          \multicolumn{1}{c|}{Min} \\ \hline
    SG0 & 6.74 & 7.06 & 8.00 & 4.68 & 2.57 & 2.72 & 3.16 & 1.73 & 4.17 & 4.19 & 5.10 & 2.91 \\ \hline
SG1 & 7.33 & 7.54 & 9.13 & 1.60 & 3.56 & 3.59 & 4.66 & 1.05 & 3.77 & 4.04 & 4.50 & 0.56 \\ \hline
SG2 & 7.64 & 8.11 & 10.06 & 3.36 & 3.68 & 3.88 & 5.56 & 1.50 & 3.96 & 4.16 & 4.66 & 1.86 \\ \hline

        \end{tabular}%
        }
        \caption{Cost details regarding Overall, Input, and Output in three levels of JSON structure usage. 11.14}
        \label{tab:cost_details_11.14}
        \end{table}
    
\begin{table}[!h]
    \centering
    \resizebox{\textwidth}{!}{%
    \begin{tabular}{|l|r|r|r|r|r|r|r|}
    \hline
    \multicolumn{1}{|c|}{} &
          \multicolumn{3}{c|}{Ave Execution Time (s)} &
          \multicolumn{3}{c|}{Ave Total Tokens} &
          \multicolumn{1}{c|}{Success} \\ \cline{2-8}
        \multicolumn{1}{|c|}{\multirow{-2}{*}{Stages}} &
          \multicolumn{1}{c|}{Overall} &
          \multicolumn{1}{c|}{Pass} &
          \multicolumn{1}{c|}{Fail} &
          \multicolumn{1}{c|}{Overall} &
          \multicolumn{1}{c|}{Pass} &
          \multicolumn{1}{c|}{Fail} &
          \multicolumn{1}{c|}{Rate (\%)} \\ \hline
    SG0 & 87.42 & 85.02 & 88.22 & 14435 & 12356 & 15128 & 0.250 \\ \hline
SG1 & 42.76 & 38.18 & 44.72 & 18014 & 17299 & 18320 & 0.300 \\ \hline
SG2 & 53.39 & 30.87 & 57.36 & 18671 & 11522 & 19933 & 0.150 \\ \hline

        \end{tabular}%
        }
    \caption{Performance metrics of passed and failed runs,including execution time, token consumption, and success rate in three levels of JSON structure usage. 11.14}
    \label{tab:performance_metrics_ave_11.14}
\end{table}
    

    \begin{table}[!h]
        \centering
        \begin{tabular}{|l|r|r|r|r|}
        \hline
        \multicolumn{1}{|c|}{Stages} &
          \multicolumn{1}{c|}{Ave} &
          \multicolumn{1}{c|}{Med} &
          \multicolumn{1}{c|}{Max} &
          \multicolumn{1}{c|}{Min} \\ \hline
    SG0 & 87.42 & 73.25 & 174.55 & 44.50 \\ \hline
SG1 & 42.76 & 43.31 & 76.44 & 7.03 \\ \hline
SG2 & 53.39 & 46.85 & 145.49 & 22.41 \\ \hline

        \end{tabular}
        \caption{Time consumption (seconds) in three levels of JSON structure usage. 11.14}
        \label{tab:time_11.14}
        \end{table}
    




\begin{figure}[!h]
    \centering
    \includegraphics[width=0.75\linewidth]{data_analysis/2024/11.14/tex/scatter_plot_11.14.pdf}
    \caption{Distribution of time and token consumption in three levels of JSON structure. Model: GPT-4o, Date: 11.14.}
    \label{fig:scatter_11.14}
\end{figure}

\begin{figure}[!h]
    \centering
    \includegraphics[width=0.75\linewidth]{data_analysis/2024/11.14/tex/bar_plot.pdf}
    \caption{Average Time and Token Consumption by different levels of JSON Usage. Model: GPT-4o, Date: 11.14. \\ \textit{The average time (top) and token consumption (bottom) are compared across three levels of JSON structure, broken down by overall performance (considering all runs), performance of only successful runs, and of only the failed runs.}}
    \label{fig:bar_11.14}
\end{figure}


\begin{table}[!h]
\centering
\resizebox{\textwidth}{!}{%
\begin{tabular}{|l|l|r|r|r|r|}
\hline
\multicolumn{1}{|c|}{Stage} & 
\multicolumn{1}{c|}{Status} & 
\multicolumn{1}{c|}{Mean} & 
\multicolumn{1}{c|}{MAD} & 
\multicolumn{1}{c|}{MAD/Mean (\%)} & 
\multicolumn{1}{c|}{StdDev/Mean (\%)} \\ \hline
\multirow{3}{*}{SG0} & Overall & 87 & 29 & 34 & 43 \\ \cline{2-6}
 & Success & 85 & 32 & 37 & 45 \\ \cline{2-6}
 & Error & 88 & 29 & 33 & 44 \\ \hline
\multirow{3}{*}{SG1} & Overall & 43 & 9 & 20 & 31 \\ \cline{2-6}
 & Success & 38 & 6 & 15 & 22 \\ \cline{2-6}
 & Error & 45 & 9 & 20 & 33 \\ \hline
\multirow{3}{*}{SG2} & Overall & 53 & 16 & 30 & 51 \\ \cline{2-6}
 & Success & 31 & 6 & 18 & 27 \\ \cline{2-6}
 & Error & 57 & 17 & 30 & 48 \\ \hline
\end{tabular}%
}
\caption{Mean Absolute Deviation Analysis for Time Consumption. 11.14}
\label{tab:mad_time_11.14}
\end{table}
 
\begin{table}[!h]
\centering
\resizebox{\textwidth}{!}{%
\begin{tabular}{|l|l|r|r|r|r|}
\hline
\multicolumn{1}{|c|}{Dataset} & 
\multicolumn{1}{c|}{Status} & 
\multicolumn{1}{c|}{Mean} & 
\multicolumn{1}{c|}{MAD} & 
\multicolumn{1}{c|}{MAD/Mean (\%)} & 
\multicolumn{1}{c|}{StdDev/Mean (\%)} \\ \hline
\multirow{3}{*}{SG0} & Overall & 14435 & 1926 & 13 & 15 \\ \cline{2-6}
 & Success & 12356 & 1881 & 15 & 20 \\ \cline{2-6}
 & Error & 15128 & 1433 & 9 & 11 \\ \hline
\multirow{3}{*}{SG1} & Overall & 18014 & 2746 & 15 & 23 \\ \cline{2-6}
 & Success & 17300 & 2106 & 12 & 17 \\ \cline{2-6}
 & Error & 18320 & 2976 & 16 & 25 \\ \hline
\multirow{3}{*}{SG2} & Overall & 18672 & 3199 & 17 & 23 \\ \cline{2-6}
 & Success & 11522 & 2584 & 22 & 33 \\ \cline{2-6}
 & Error & 19933 & 2064 & 10 & 14 \\ \hline
\end{tabular}%
}
\caption{Mean Absolute Deviation Analysis for Tokens Consumption. 11.14}
\label{tab:mad_tokens_11.14}
\end{table}


\begin{figure}[!h]
    \centering
    \includegraphics[width=0.75\linewidth]{data_analysis/2024/11.14/tex/mad_visualization.pdf}
    \caption{Mean Absolute Deviation of Time and Token Usage Model: GPT-4o, Date: 11.14.}
    \label{fig:mad_vis_11.14}
\end{figure}

\begin{figure}[!h]
    \centering
    \includegraphics[width=0.75\linewidth]{data_analysis/2024/11.14/tex/relative_variability_time.pdf}
    \caption{Relative variability (MAD/Mean) of Time Comsumption Model: GPT-4o, Date: 11.14.}
    \label{fig:mad_vis_11.14}
\end{figure}

\begin{figure}[!h]
    \centering
    \includegraphics[width=0.75\linewidth]{data_analysis/2024/11.14/tex/relative_variability_tokens.pdf}
    \caption{Relative variability (MAD/Mean) of Token Usage Model: GPT-4o, Date: 11.14.}
    \label{fig:mad_vis_11.14}
\end{figure}
\newpage
\FloatBarrier
\newpage
\subsection{2024.11.28, GPT-4o-mini}
\FloatBarrier
\input{data_analysis/2024/11.28/tex/tab_token_details_11.28}

    \begin{table}[!h]
        \centering
        \resizebox{\textwidth}{!}{%
        \begin{tabular}{|l|r|r|r|r|r|r|r|}
        \hline
        \multicolumn{1}{|c|}{} &
          \multicolumn{3}{c|}{Execution Time (s)} &
          \multicolumn{3}{c|}{Token Consumption} &
          \multicolumn{1}{c|}{Success} \\ \cline{2-8}
        \multicolumn{1}{|c|}{\multirow{-2}{*}{Stages}} &
          \multicolumn{1}{c|}{Ave} &
          \multicolumn{1}{c|}{Max} &
          \multicolumn{1}{c|}{Min} &
          \multicolumn{1}{c|}{Ave} &
          \multicolumn{1}{c|}{Max} &
          \multicolumn{1}{c|}{Min} &
          \multicolumn{1}{c|}{Rate (\%)} \\ \hline
    DP & 91.24 & 149.04 & 60.04 & 21693 & 36578 & 17390 & 0.900 \\ \hline
MC & 12.57 & 34.66 & 3.02 & 5843 & 12801 & 519 & 0.750 \\ \hline
SG & 73.93 & 114.93 & 15.25 & 27406 & 33098 & 5172 & 0.100 \\ \hline

        \end{tabular}%
        }
        \caption{Performance metrics including execution time, token consumption, and success rate in three stages. Model: GPT-4o-mini, Date: 11.28}
        \label{tab:performance_metrics_11.28}
        \end{table}
    

    \begin{table}[!h]
        \centering
        \resizebox{\textwidth}{!}{%
        \begin{tabular}{|l|r|r|r|r|r|r|r|r|r|r|r|r|}
        \hline
        \multicolumn{1}{|c|}{} &
          \multicolumn{4}{c|}{Overall Cost (USD cents)} &
          \multicolumn{4}{c|}{Input Cost (USD cents)} &
          \multicolumn{4}{c|}{Output Cost (USD cents)} \\ \cline{2-13}
        \multicolumn{1}{|c|}{\multirow{-2}{*}{Stages}} &
          \multicolumn{1}{c|}{Ave} &
          \multicolumn{1}{c|}{Med} &
          \multicolumn{1}{c|}{Max} &
          \multicolumn{1}{c|}{Min} &
          \multicolumn{1}{c|}{Ave} &
          \multicolumn{1}{c|}{Med} &
          \multicolumn{1}{c|}{Max} &
          \multicolumn{1}{c|}{Min} &
          \multicolumn{1}{c|}{Ave} &
          \multicolumn{1}{c|}{Med} &
          \multicolumn{1}{c|}{Max} &
          \multicolumn{1}{c|}{Min} \\ \hline
    DP & 0.55 & 0.50 & 0.96 & 0.43 & 0.25 & 0.21 & 0.42 & 0.20 & 0.30 & 0.29 & 0.55 & 0.22 \\ \hline
MC & 0.12 & 0.08 & 0.25 & 0.02 & 0.08 & 0.05 & 0.17 & 0.00 & 0.05 & 0.03 & 0.10 & 0.01 \\ \hline
SG & 0.68 & 0.74 & 0.85 & 0.13 & 0.32 & 0.36 & 0.38 & 0.06 & 0.35 & 0.38 & 0.47 & 0.07 \\ \hline

        \end{tabular}%
        }
        \caption{Cost details regarding Overall, Input, and Output in three stages. Model: GPT-4o-mini, Date: 11.28}
        \label{tab:cost_details_11.28}
        \end{table}
    

    \begin{table}[!h]
        \centering
        \resizebox{\textwidth}{!}{%
        \begin{tabular}{|l|r|r|r|r|r|r|r|}
        \hline
        \multicolumn{1}{|c|}{} &
          \multicolumn{3}{c|}{Ave Execution Time (s)} &
          \multicolumn{3}{c|}{Ave Total Tokens} &
          \multicolumn{1}{c|}{Success} \\ \cline{2-8}
        \multicolumn{1}{|c|}{\multirow{-2}{*}{Stages}} &
          \multicolumn{1}{c|}{Overall} &
          \multicolumn{1}{c|}{Pass} &
          \multicolumn{1}{c|}{Fail} &
          \multicolumn{1}{c|}{Overall} &
          \multicolumn{1}{c|}{Pass} &
          \multicolumn{1}{c|}{Fail} &
          \multicolumn{1}{c|}{Rate (\%)} \\ \hline
    DP & 91.24 & 91.03 & 93.17 & 21693 & 21023 & 27723 & 0.900 \\ \hline
MC & 12.57 & 9.08 & 23.03 & 5843 & 4413 & 10131 & 0.750 \\ \hline
SG & 73.93 & 15.75 & 80.39 & 27406 & 5213 & 29872 & 0.100 \\ \hline

        \end{tabular}%
        }
        \caption{Performance metrics of passed and failed runs,including execution time, token consumption, and success rate in three stages. Model: GPT-4o-mini, Date: 11.28}
        \label{tab:performance_metrics_ave_11.28}
        \end{table}
    

    \begin{table}[!h]
        \centering
        \begin{tabular}{|l|r|r|r|r|}
        \hline
        \multicolumn{1}{|c|}{Stages} &
          \multicolumn{1}{c|}{Ave} &
          \multicolumn{1}{c|}{Med} &
          \multicolumn{1}{c|}{Max} &
          \multicolumn{1}{c|}{Min} \\ \hline
    DP & 91.24 & 85.82 & 149.04 & 60.04 \\ \hline
MC & 12.57 & 8.89 & 34.66 & 3.02 \\ \hline
SG & 73.93 & 76.27 & 114.93 & 15.25 \\ \hline

        \end{tabular}
        \caption{Time consumption (seconds) in three stages. Model: GPT-4o-mini, Date: 11.28}
        \label{tab:time_11.28}
        \end{table}
    

\begin{figure}[!h]
    \centering
    \includegraphics[width=0.75\linewidth]{data_analysis/2024/11.28/tex/scatter_plot_11.28.pdf}
    \caption{Distribution of time and token consumption in three tasks. Model: GPT-4o-mini, Date: 11.28.}
    \label{fig:scatter_11.28}
\end{figure}

\begin{figure}[!h]
    \centering
    \includegraphics[width=0.75\linewidth]{data_analysis/2024/11.28/tex/bar_plot_11.28.pdf}
    \caption{Average Time and Token Consumption by Task Status. Model: GPT-4o-mini, Date: 11.28. \\ \textit{The average time (top) and token consumption (bottom) are compared across three stages, broken down by overall performance (considering all runs), performance of only successful runs, and of only the failed runs.}}
    \label{fig:bar_11.28}
\end{figure}
\FloatBarrier
\newpage



% \FloatBarrier
% \newpage
\subsection{Results: 2025.01.15, Model: phi4:14b}
\FloatBarrier
\input{data_analysis/2025/01.15/tex/tab_token_details_phi4:14b_01.15}

    \begin{table}[!h]
        \centering
        \resizebox{\textwidth}{!}{%
        \begin{tabular}{|l|r|r|r|r|r|r|r|}
        \hline
        \multicolumn{1}{|c|}{} &
          \multicolumn{3}{c|}{Execution Time (s)} &
          \multicolumn{3}{c|}{Token Consumption} &
          \multicolumn{1}{c|}{} \\ \cline{2-7}
        \multicolumn{1}{|c|}{\multirow{-2}{*}{Stages}} &
          \multicolumn{1}{c|}{Ave} &
          \multicolumn{1}{c|}{Max} &
          \multicolumn{1}{c|}{Min} &
          \multicolumn{1}{c|}{Ave} &
          \multicolumn{1}{c|}{Max} &
          \multicolumn{1}{c|}{Min} &
          \multicolumn{1}{|c|}{\multirow{-2}{*}{\shortstack{Success\\Rate}}} \\ \hline
    DP & 47.97 & 98.57 & 32.06 & 15220 & 28556 & 11309 & 0.500 \\ \hline
MC & 11.87 & 41.03 & 3.83 & 2149 & 8731 & 674 & 0.967 \\ \hline

        \end{tabular}%
        }
        \caption{Performance metrics including execution time, token consumption, and success rate in three stages. Model: phi4:14b, Date: 01.15}
        \label{tab:performance_metrics_phi4:14b_01.15}
        \end{table}
    
\input{data_analysis/2025/01.15/tex/tab_cost_details_phi4:14b_01.15}

    \begin{table}[!h]
        \centering
        \resizebox{\textwidth}{!}{%
        \begin{tabular}{|l|r|r|r|r|r|r|r|}
        \hline
        \multicolumn{1}{|c|}{} &
          \multicolumn{3}{c|}{Ave Execution Time (s)} &
          \multicolumn{3}{c|}{Ave Total Tokens} &
          \multicolumn{1}{c|}{} \\ \cline{2-7}
        \multicolumn{1}{|c|}{\multirow{-2}{*}{Stages}} &
          \multicolumn{1}{c|}{Overall} &
          \multicolumn{1}{c|}{Pass} &
          \multicolumn{1}{c|}{Fail} &
          \multicolumn{1}{c|}{Overall} &
          \multicolumn{1}{c|}{Pass} &
          \multicolumn{1}{c|}{Fail} &
          \multicolumn{1}{|c|}{\multirow{-2}{*}{\shortstack{Success\\Rate}}} \\ \hline
    DP & 47.97 & 39.03 & 56.91 & 15220 & 12867 & 17573 & 0.500 \\ \hline
MC & 11.87 & 11.16 & 32.57 & 2149 & 1922 & 8731 & 0.967 \\ \hline

        \end{tabular}%
        }
        \caption{Performance metrics of passed and failed runs,including execution time, token consumption, and success rate in three stages. Model: phi4:14b, Date: 01.15}
        \label{tab:performance_metrics_ave_phi4:14b_01.15}
        \end{table}
    

    \begin{table}[!h]
        \centering
        \begin{tabular}{|l|r|r|r|r|}
        \hline
        \multicolumn{1}{|c|}{Stages} &
          \multicolumn{1}{c|}{Ave} &
          \multicolumn{1}{c|}{Med} &
          \multicolumn{1}{c|}{Max} &
          \multicolumn{1}{c|}{Min} \\ \hline
    DP & 47.97 & 43.81 & 98.57 & 32.06 \\ \hline
MC & 11.87 & 5.13 & 41.03 & 3.83 \\ \hline

        \end{tabular}
        \caption{Time consumption (seconds) in three stages. Model: phi4:14b, Date: 01.15}
        \label{tab:time_phi4:14b_01.15}
        \end{table}
    
\newpage
\begin{figure}[!h]
    \centering
    \includegraphics[width=0.75\linewidth]{data_analysis/2025/01.15/tex/scatter_plot_phi4:14b_01.15.pdf}
    \caption{Distribution of time and token consumption in three tasks. Model: phi4:14b, Date: 01.15.}
    \label{fig:scatter_phi4:14b_01.15}
\end{figure}

\begin{figure}[!h]
    \centering
    \includegraphics[width=0.75\linewidth]{data_analysis/2025/01.15/tex/bar_plot_phi4:14b_01.15.pdf}
    \caption{Average Time and Token Consumption by Task Status. Model: phi4:14b, Date: 01.15. \\ \textit{The average time (top) and token consumption (bottom) are compared across three stages, broken down by overall performance (considering all runs), performance of only successful runs, and of only the failed runs.}}
    \label{fig:bar_phi4:14b_01.15}
\end{figure}



\newpage
\FloatBarrier
\subsection{Results: 2025.01.15, Model: llama3.1:8b}


    \begin{table}[!h]
        \centering
        \resizebox{\textwidth}{!}{%
        \begin{tabular}{|l|r|r|r|r|r|r|r|r|r|r|r|r|}
        \hline
        \multicolumn{1}{|c|}{} &
          \multicolumn{4}{c|}{Overall Token} &
          \multicolumn{4}{c|}{Input Token} &
          \multicolumn{4}{c|}{Output Token} \\ \cline{2-13}
        \multicolumn{1}{|c|}{\multirow{-2}{*}{Stages}} &
          \multicolumn{1}{c|}{Ave} &
          \multicolumn{1}{c|}{Med} &
          \multicolumn{1}{c|}{Max} &
          \multicolumn{1}{c|}{Min} &
          \multicolumn{1}{c|}{Ave} &
          \multicolumn{1}{c|}{Med} &
          \multicolumn{1}{c|}{Max} &
          \multicolumn{1}{c|}{Min} &
          \multicolumn{1}{c|}{Ave} &
          \multicolumn{1}{c|}{Med} &
          \multicolumn{1}{c|}{Max} &
          \multicolumn{1}{c|}{Min} \\ \hline
    DP & 16869 & 16563 & 35068 & 2076 & 13469 & 13143 & 26029 & 1782 & 3400 & 3420 & 9039 & 294 \\ \hline
MC & 4636 & 5805 & 8555 & 523 & 3797 & 4880 & 6960 & 389 & 838 & 908 & 1595 & 134 \\ \hline

        \end{tabular}%
        }
        \caption{Token consumption details regarding Overall, Input, and Output in three stages. Model: llama3.1:8b, Date: 01.15}
        \label{tab:token_details_llama3.1:8b_01.15}
        \end{table}
    

    \begin{table}[!h]
        \centering
        \resizebox{\textwidth}{!}{%
        \begin{tabular}{|l|r|r|r|r|r|r|r|}
        \hline
        \multicolumn{1}{|c|}{} &
          \multicolumn{3}{c|}{Execution Time (s)} &
          \multicolumn{3}{c|}{Token Consumption} &
          \multicolumn{1}{c|}{} \\ \cline{2-7}
        \multicolumn{1}{|c|}{\multirow{-2}{*}{Stages}} &
          \multicolumn{1}{c|}{Ave} &
          \multicolumn{1}{c|}{Max} &
          \multicolumn{1}{c|}{Min} &
          \multicolumn{1}{c|}{Ave} &
          \multicolumn{1}{c|}{Max} &
          \multicolumn{1}{c|}{Min} &
          \multicolumn{1}{|c|}{\multirow{-2}{*}{\shortstack{Success\\Rate}}} \\ \hline
    DP & 35.44 & 92.65 & 3.11 & 16869 & 35068 & 2076 & 0.200 \\ \hline
MC & 8.77 & 16.66 & 1.28 & 4636 & 8555 & 523 & 0.633 \\ \hline

        \end{tabular}%
        }
        \caption{Performance metrics including execution time, token consumption, and success rate in three stages. Model: llama3.1:8b, Date: 01.15}
        \label{tab:performance_metrics_llama3.1:8b_01.15}
        \end{table}
    
\input{data_analysis/2025/01.15/tex/tab_cost_details_llama3.1:8b_01.15}

    \begin{table}[!h]
        \centering
        \resizebox{\textwidth}{!}{%
        \begin{tabular}{|l|r|r|r|r|r|r|r|}
        \hline
        \multicolumn{1}{|c|}{} &
          \multicolumn{3}{c|}{Ave Execution Time (s)} &
          \multicolumn{3}{c|}{Ave Total Tokens} &
          \multicolumn{1}{c|}{} \\ \cline{2-7}
        \multicolumn{1}{|c|}{\multirow{-2}{*}{Stages}} &
          \multicolumn{1}{c|}{Overall} &
          \multicolumn{1}{c|}{Pass} &
          \multicolumn{1}{c|}{Fail} &
          \multicolumn{1}{c|}{Overall} &
          \multicolumn{1}{c|}{Pass} &
          \multicolumn{1}{c|}{Fail} &
          \multicolumn{1}{|c|}{\multirow{-2}{*}{\shortstack{Success\\Rate}}} \\ \hline
    DP & 35.44 & 36.30 & 35.23 & 16869 & 17621 & 16681 & 0.200 \\ \hline
MC & 8.77 & 6.28 & 13.09 & 4636 & 3119 & 7256 & 0.633 \\ \hline

        \end{tabular}%
        }
        \caption{Performance metrics of passed and failed runs,including execution time, token consumption, and success rate in three stages. Model: llama3.1:8b, Date: 01.15}
        \label{tab:performance_metrics_ave_llama3.1:8b_01.15}
        \end{table}
    

    \begin{table}[!h]
        \centering
        \begin{tabular}{|l|r|r|r|r|}
        \hline
        \multicolumn{1}{|c|}{Stages} &
          \multicolumn{1}{c|}{Ave} &
          \multicolumn{1}{c|}{Med} &
          \multicolumn{1}{c|}{Max} &
          \multicolumn{1}{c|}{Min} \\ \hline
    DP & 35.44 & 35.41 & 92.65 & 3.11 \\ \hline
MC & 8.77 & 9.88 & 16.66 & 1.28 \\ \hline

        \end{tabular}
        \caption{Time consumption (seconds) in three stages. Model: llama3.1:8b, Date: 01.15}
        \label{tab:time_llama3.1:8b_01.15}
        \end{table}
    
\newpage
\begin{figure}[!h]
    \centering
    \includegraphics[width=0.75\linewidth]{data_analysis/2025/01.15/tex/scatter_plot_llama3.1:8b_01.15.pdf}
    \caption{Distribution of time and token consumption in three tasks. Model: llama3.1:8b, Date: 01.15.}
    \label{fig:scatter_llama3.1:8b_01.15}
\end{figure}

\begin{figure}[!h]
    \centering
    \includegraphics[width=0.75\linewidth]{data_analysis/2025/01.15/tex/bar_plot_llama3.1:8b_01.15.pdf}
    \caption{Average Time and Token Consumption by Task Status. Model: llama3.1:8b, Date: 01.15. \\ \textit{The average time (top) and token consumption (bottom) are compared across three stages, broken down by overall performance (considering all runs), performance of only successful runs, and of only the failed runs.}}
    \label{fig:bar_llama3.1:8b_01.15}
\end{figure}
% \FloatBarrier
% \newpage
\subsection{2025.01.23, qwen2.5-coder:14b}
\FloatBarrier

    \begin{table}[!h]
        \centering
        \resizebox{\textwidth}{!}{%
        \begin{tabular}{|l|r|r|r|r|r|r|r|r|r|r|r|r|}
        \hline
        \multicolumn{1}{|c|}{} &
          \multicolumn{4}{c|}{Overall Token} &
          \multicolumn{4}{c|}{Input Token} &
          \multicolumn{4}{c|}{Output Token} \\ \cline{2-13}
        \multicolumn{1}{|c|}{\multirow{-2}{*}{Stages}} &
          \multicolumn{1}{c|}{Ave} &
          \multicolumn{1}{c|}{Med} &
          \multicolumn{1}{c|}{Max} &
          \multicolumn{1}{c|}{Min} &
          \multicolumn{1}{c|}{Ave} &
          \multicolumn{1}{c|}{Med} &
          \multicolumn{1}{c|}{Max} &
          \multicolumn{1}{c|}{Min} &
          \multicolumn{1}{c|}{Ave} &
          \multicolumn{1}{c|}{Med} &
          \multicolumn{1}{c|}{Max} &
          \multicolumn{1}{c|}{Min} \\ \hline
    DP & 16929 & 16088 & 27965 & 12076 & 13991 & 13478 & 22244 & 10068 & 2937 & 2652 & 5745 & 1434 \\ \hline
MC & 6106 & 7162 & 9488 & 2420 & 4848 & 5640 & 7545 & 1942 & 1258 & 1505 & 2058 & 391 \\ \hline

        \end{tabular}%
        }
        \caption{Token consumption details regarding Overall, Input, and Output in three stages. Model: qwen2.5-coder:14b, Date: 01.23}
        \label{tab:token_details_qwen2.5-coder:14b_01.23}
        \end{table}
    
\input{data_analysis/2025/01.23/tex/tab_performance_metrics_qwen2.5-coder:14b_01.23}
\input{data_analysis/2025/01.23/tex/tab_cost_details_qwen2.5-coder:14b_01.23}

    \begin{table}[!h]
        \centering
        \resizebox{\textwidth}{!}{%
        \begin{tabular}{|l|r|r|r|r|r|r|r|}
        \hline
        \multicolumn{1}{|c|}{} &
          \multicolumn{3}{c|}{Ave Execution Time (s)} &
          \multicolumn{3}{c|}{Ave Total Tokens} &
          \multicolumn{1}{c|}{} \\ \cline{2-7}
        \multicolumn{1}{|c|}{\multirow{-2}{*}{Stages}} &
          \multicolumn{1}{c|}{Overall} &
          \multicolumn{1}{c|}{Pass} &
          \multicolumn{1}{c|}{Fail} &
          \multicolumn{1}{c|}{Overall} &
          \multicolumn{1}{c|}{Pass} &
          \multicolumn{1}{c|}{Fail} &
          \multicolumn{1}{|c|}{\multirow{-2}{*}{\shortstack{Success\\Rate}}} \\ \hline
    DP & 52.46 & 51.36 & 52.93 & 16929 & 14582 & 17935 & 0.300 \\ \hline
MC & 22.16 & 16.67 & 29.34 & 6106 & 4462 & 8257 & 0.567 \\ \hline

        \end{tabular}%
        }
        \caption{Performance metrics of passed and failed runs,including execution time, token consumption, and success rate in three stages. Model: qwen2.5-coder:14b, Date: 01.23}
        \label{tab:performance_metrics_ave_qwen2.5-coder:14b_01.23}
        \end{table}
    

    \begin{table}[!h]
        \centering
        \begin{tabular}{|l|r|r|r|r|}
        \hline
        \multicolumn{1}{|c|}{Stages} &
          \multicolumn{1}{c|}{Ave} &
          \multicolumn{1}{c|}{Med} &
          \multicolumn{1}{c|}{Max} &
          \multicolumn{1}{c|}{Min} \\ \hline
    DP & 52.46 & 46.11 & 138.04 & 26.02 \\ \hline
MC & 22.16 & 25.55 & 36.77 & 7.23 \\ \hline

        \end{tabular}
        \caption{Time consumption (seconds) in three stages. Model: qwen2.5-coder:14b, Date: 01.23}
        \label{tab:time_qwen2.5-coder:14b_01.23}
        \end{table}
    
\newpage
\begin{figure}[!h]
    \centering
    \includegraphics[width=0.75\linewidth]{data_analysis/2025/01.23/tex/scatter_plot_qwen2.5-coder:14b_01.23.pdf}
    \caption{Distribution of time and token consumption in three tasks. Model: qwen2.5-coder:14b, Date: 01.23.}
    \label{fig:scatter_qwen2.5-coder:14b_01.23}
\end{figure}

\begin{figure}[!h]
    \centering
    \includegraphics[width=0.75\linewidth]{data_analysis/2025/01.23/tex/bar_plot_qwen2.5-coder:14b_01.23.pdf}
    \caption{Average Time and Token Consumption by Task Status. Model: qwen2.5-coder:14b, Date: 01.23. \\ \textit{The average time (top) and token consumption (bottom) are compared across three stages, broken down by overall performance (considering all runs), performance of only successful runs, and of only the failed runs.}}
    \label{fig:bar_qwen2.5-coder:14b_01.23}
\end{figure}



\newpage
\subsection{ 2025.01.23, Model: deepseek-r1:14b}
In this test, we used Deepseek's \texttt{deepseek-r1:14b}, which is a reasoning LLM in their reasoning series of LLMs. Similar to OpenAI's \texttt{o1} series, it trades off speed for more powerful reasoning ability. It also outputs the reasoning process before the real answer to the question. A typical output is shown below \ref{fig:deepseekr1_example}. Although the model is powerful at its reasoning capability according to recent studies and tests, our own test results that are based on using its direct output as engineering code are not outstanding. Perhaps the reasoning model series can be excellent for providing helpful insights into human-in-the-loop engineering tasks instead.
\begin{figure}[!h]
    \centering
\includegraphics[width=0.75\linewidth]{data_analysis/2025/01.23/tex/deepseekr1_example.png}
\caption{An example output of \texttt{deepseek-r1:14b}.}
\label{fig:deepseekr1_example}
\end{figure}
\input{data_analysis/2025/01.23/tex/tab_token_details_deepseek-r1:14b_01.23}

    \begin{table}[!h]
        \centering
        \resizebox{\textwidth}{!}{%
        \begin{tabular}{|l|r|r|r|r|r|r|r|}
        \hline
        \multicolumn{1}{|c|}{} &
          \multicolumn{3}{c|}{Execution Time (s)} &
          \multicolumn{3}{c|}{Token Consumption} &
          \multicolumn{1}{c|}{} \\ \cline{2-7}
        \multicolumn{1}{|c|}{\multirow{-2}{*}{Stages}} &
          \multicolumn{1}{c|}{Ave} &
          \multicolumn{1}{c|}{Max} &
          \multicolumn{1}{c|}{Min} &
          \multicolumn{1}{c|}{Ave} &
          \multicolumn{1}{c|}{Max} &
          \multicolumn{1}{c|}{Min} &
          \multicolumn{1}{|c|}{\multirow{-2}{*}{\shortstack{Success\\Rate}}} \\ \hline
    DP & 204.56 & 373.34 & 87.83 & 31304 & 57981 & 13742 & 0.267 \\ \hline
MC & 58.05 & 84.15 & 13.24 & 8589 & 11289 & 1248 & 0.333 \\ \hline

        \end{tabular}%
        }
        \caption{Performance metrics including execution time, token consumption, and success rate in three stages. Model: deepseek-r1:14b, Date: 01.23}
        \label{tab:performance_metrics_deepseek-r1:14b_01.23}
        \end{table}
    
\input{data_analysis/2025/01.23/tex/tab_cost_details_deepseek-r1:14b_01.23}

    \begin{table}[!h]
        \centering
        \resizebox{\textwidth}{!}{%
        \begin{tabular}{|l|r|r|r|r|r|r|r|}
        \hline
        \multicolumn{1}{|c|}{} &
          \multicolumn{3}{c|}{Ave Execution Time (s)} &
          \multicolumn{3}{c|}{Ave Total Tokens} &
          \multicolumn{1}{c|}{} \\ \cline{2-7}
        \multicolumn{1}{|c|}{\multirow{-2}{*}{Stages}} &
          \multicolumn{1}{c|}{Overall} &
          \multicolumn{1}{c|}{Pass} &
          \multicolumn{1}{c|}{Fail} &
          \multicolumn{1}{c|}{Overall} &
          \multicolumn{1}{c|}{Pass} &
          \multicolumn{1}{c|}{Fail} &
          \multicolumn{1}{|c|}{\multirow{-2}{*}{\shortstack{Success\\Rate}}} \\ \hline
    DP & 204.56 & 274.67 & 179.07 & 31304 & 40695 & 27890 & 0.267 \\ \hline
MC & 58.05 & 46.12 & 64.02 & 8589 & 6435 & 9667 & 0.333 \\ \hline

        \end{tabular}%
        }
        \caption{Performance metrics of passed and failed runs,including execution time, token consumption, and success rate in three stages. Model: deepseek-r1:14b, Date: 01.23}
        \label{tab:performance_metrics_ave_deepseek-r1:14b_01.23}
        \end{table}
    

    \begin{table}[!h]
        \centering
        \begin{tabular}{|l|r|r|r|r|}
        \hline
        \multicolumn{1}{|c|}{Stages} &
          \multicolumn{1}{c|}{Ave} &
          \multicolumn{1}{c|}{Med} &
          \multicolumn{1}{c|}{Max} &
          \multicolumn{1}{c|}{Min} \\ \hline
    DP & 204.56 & 184.48 & 373.34 & 87.83 \\ \hline
MC & 58.05 & 58.94 & 84.15 & 13.24 \\ \hline

        \end{tabular}
        \caption{Time consumption (seconds) in three stages. Model: deepseek-r1:14b, Date: 01.23}
        \label{tab:time_deepseek-r1:14b_01.23}
        \end{table}
    
\newpage
\begin{figure}[!h]
    \centering
    \includegraphics[width=0.75\linewidth]{data_analysis/2025/01.23/tex/scatter_plot_deepseek-r1:14b_01.23.pdf}
    \caption{Distribution of time and token consumption in three tasks. Model: deepseek-r1:14b, Date: 01.23.}
    \label{fig:scatter_deepseek-r1:14b_01.23}
\end{figure}

\begin{figure}[!h]
    \centering
    \includegraphics[width=0.75\linewidth]{data_analysis/2025/01.23/tex/bar_plot_deepseek-r1:14b_01.23.pdf}
    \caption{Average Time and Token Consumption by Task Status. Model: deepseek-r1:14b, Date: 01.23. \\ \textit{The average time (top) and token consumption (bottom) are compared across three stages, broken down by overall performance (considering all runs), performance of only successful runs, and of only the failed runs.}}
    \label{fig:bar_deepseek-r1:14b_01.23}
\end{figure}



\subsection{Results: 2025.01.15/23, Models: 4 Open-source Models}
There were 4 subsections commented out: \\ {Results: 2025.01.15, Model: phi4:14b}, \\{Results: 2025.01.15, Model: llama3.1:8b}, \\{Results: 2025.01.23, Model: qwen2.5-coder:14b}, \\{Results: 2025.01.23, Model: deepseek-r1:14b}

They were only DP and MC, now they are merged into the Results: 2025.03.18 related sections. 

\FloatBarrier

\newpage
\subsection{Discussion: 2025.02.04, Output format handling issues with open-source models}

Open-source models' output can break the workflow by making faults in the middle and completely misleading the follow-up steps in the flow by \textbf{case 1).} incorrect/unparsable format of output (especially when i ask for merely code blocks while it outputs with explaninary text), \textbf{case 2).} wrong information or important information skipped, with right format, causing false positive results, \textbf{case 3).} not following the instructions in the prompt, not understanding the output/dataset path so that modifications happen in the directory it shouldnt touch. The SG stage is especially vulnerable to the issue case 2. DP is sometimes confused by issue case 3. 

Here are the cases with examples:
\subsubsection*{Case 1 - incorrect/unparsable format of output:}

This is very common in all stages, especially in DP and SG, because they have complex/multi-step code generation. For example, when repeatedly debugging one step and getting errors, the LLM tends to instruct me something about its code, or it outputs only the code snippets of the needed correction instead of the full code. 

It's fine to output explanation text along with (before or after) the code block containing complete code, because the output parsing will pick out the \texttt{```python/cpp/ino/json '''} code block and run it. What's not dealable with is \textbf{1)} mixing explanation text with code blocks, when they are alternate, code snippet + explanation by code snippet + explanation, as shown in the example \ref{fig:code-text-alternate}; or similarly, giving text as the majority, code snippets as the exmaple of its explanation, dispersive and incomplete code blocks, as examplified in \ref{fig:code-in-text}; \textbf{2)} Giving multiple solutions in the code, telling user to choose one accrodingly. Typically with alternative code pieces (method invocations, etc.) commented out, as shown in \ref{fig:multiple-solutions}. \textbf{To conclude, it doesnt work when there is no single one code block containing complete code in the output, while this is always required in the prompt.}



\begin{figure}[!h]
    \centering
    \includegraphics[width=0.99\linewidth]{data_analysis/2025/02.04/code-text-alternate.png}
    \caption{Example of code and text being alternate.}
    \label{fig:code-text-alternate}
\end{figure}


\begin{figure}[!h]
    \centering
    \includegraphics[width=0.99\linewidth]{data_analysis/2025/02.04/code-in-text.png}
    \caption{Example of code given under/inside explaninary text.}
    \label{fig:code-in-text}
\end{figure}


\begin{figure}[!h]
    \centering
    \includegraphics[width=0.99\linewidth]{data_analysis/2025/02.04/multiple-solutions.png}
    \caption{Example of given solution examples which requires user to choose one.}
    \label{fig:multiple-solutions}
\end{figure}


\subsubsection*{Case 2 - bad code with missing operations passes the test in a false positive way:}

\textcolor{red}{This was actually a bug in the error handing for specification filing and now it's fixed: } \textit{I was mentioning that when constructing the skecth specification for skecth generation, bad spec list with missing information leads to something-else sketch code, and that allows false positve sketch generation (under wrong or faulty specification, the returen code is compilable and passing our test, but the skecth is not practically usable, e.g., a sketch simply invokes color senser and has nothing to do with the machine leaning model). }

\textbf{But the problems with usability of compilable sketch still exists. I found that some of the compiled \texttt{.ino} files indeed has no response when flashed to Arduino, and this needs to be attach importance to and addressed. I think this is where we would use our testbed.} 



This also happens in DP, due to it's multi-step and long process natures of one whole task. For example, the image \ref{fig:wrong-path} shows the case that the code reads the given path correctly, but doesnt save the processed dataset, while reporting the dataset is updated. This causes false positive in the current step, and breaking the follow-up steps.The next step would seek for a path that doesn't exist, and the whole flow would be broken.

\begin{figure}[!h]
    \centering
    \includegraphics[width=0.99\linewidth]{data_analysis/2025/02.04/case2-path.png}
    \caption{Example of unsaved but reported data processing, causing false positive in the current step, and breaking the follow-up steps.}
    \label{fig:wrong-path}
\end{figure}



\subsubsection*{Case 3 - messing up directories:}


All the data processing related mediate artefacts, including \texttt{*.csv} files and folders like \texttt{processed} should only be placed under \texttt{data/fruit\_to\_emoji/playground}. And the artefacts about model conversion like \texttt{*.tflite} should be under \texttt{models/fruit\_to\_emoji/tflite\_model}. As shown in the image \ref{fig:mess-dir}, all the mediate folders and files before which there's the red bar are located in wrong plcaes. They should either be under \texttt{data/fruit\_to\_emoji/playground} which is rectangled in green, or \texttt{models/fruit\_to\_emoji/tflite\_model}.   

This also includes the naming of the files, especially the converted model, the name is predefined as \texttt{model\_quant\_int8.tflite} in our code, but e.g. under the second red bar in \ref{fig:mess-dir}, it's named as \texttt{converted\_model.tflite}, as well as put in the wrong directory.

This happens directly because the path variables defined in the output code do not match the specificatio in the prompt. 

\begin{figure}[!h]
    \centering
    \includegraphics[width=0.85\linewidth]{data_analysis/2025/02.04/mess-dir.png}
    \caption{Results of directory mess-up.}
    \label{fig:mess-dir}
\end{figure}
 



\clearpage 

\newpage
\subsection{Results: 2025.03.18, Model: phi4:14b}
\textcolor{red}{Note: The results of DP and MC are from 2025.01.15, and the results of SG are from 2025.03.18.
}
\FloatBarrier

    \begin{table}[!h]
        \centering
        \resizebox{\textwidth}{!}{%
        \begin{tabular}{|l|r|r|r|r|r|r|r|r|r|r|r|r|}
        \hline
        \multicolumn{1}{|c|}{} &
          \multicolumn{4}{c|}{Overall Token} &
          \multicolumn{4}{c|}{Input Token} &
          \multicolumn{4}{c|}{Output Token} \\ \cline{2-13}
        \multicolumn{1}{|c|}{\multirow{-2}{*}{Stages}} &
          \multicolumn{1}{c|}{Ave} &
          \multicolumn{1}{c|}{Med} &
          \multicolumn{1}{c|}{Max} &
          \multicolumn{1}{c|}{Min} &
          \multicolumn{1}{c|}{Ave} &
          \multicolumn{1}{c|}{Med} &
          \multicolumn{1}{c|}{Max} &
          \multicolumn{1}{c|}{Min} &
          \multicolumn{1}{c|}{Ave} &
          \multicolumn{1}{c|}{Med} &
          \multicolumn{1}{c|}{Max} &
          \multicolumn{1}{c|}{Min} \\ \hline
    DP & 15220 & 14719 & 28556 & 11309 & 12031 & 11350 & 22110 & 8967 & 3189 & 2928 & 6446 & 2131 \\ \hline
MC & 2149 & 767 & 8731 & 674 & 1463 & 389 & 6561 & 389 & 686 & 378 & 2170 & 285 \\ \hline
SG & 16396 & 16349 & 20328 & 6830 & 11176 & 11046 & 14162 & 4902 & 5220 & 5284 & 6166 & 1928 \\ \hline

        \end{tabular}%
        }
        \caption{Token consumption details regarding Overall, Input, and Output in three stages. Model: phi4:14b, Date: 03.18}
        \label{tab:token_details_phi4:14b_03.18}
        \end{table}
    
\input{data_analysis/2025/03.18/tex/tab_performance_metrics_phi4:14b_03.18}
% \input{data_analysis/2025/03.18/tex/tab_cost_details_phi4:14b_03.18}

    \begin{table}[!h]
        \centering
        \resizebox{\textwidth}{!}{%
        \begin{tabular}{|l|r|r|r|r|r|r|r|}
        \hline
        \multicolumn{1}{|c|}{} &
          \multicolumn{3}{c|}{Ave Execution Time (s)} &
          \multicolumn{3}{c|}{Ave Total Tokens} &
          \multicolumn{1}{c|}{} \\ \cline{2-7}
        \multicolumn{1}{|c|}{\multirow{-2}{*}{Stages}} &
          \multicolumn{1}{c|}{Overall} &
          \multicolumn{1}{c|}{Pass} &
          \multicolumn{1}{c|}{Fail} &
          \multicolumn{1}{c|}{Overall} &
          \multicolumn{1}{c|}{Pass} &
          \multicolumn{1}{c|}{Fail} &
          \multicolumn{1}{|c|}{\multirow{-2}{*}{\shortstack{Success\\Rate}}} \\ \hline
    DP & 47.97 & 39.03 & 56.91 & 15220 & 12867 & 17573 & 0.500 \\ \hline
MC & 11.87 & 11.16 & 32.57 & 2149 & 1922 & 8731 & 0.967 \\ \hline
SG & 97.60 & 30.36 & 99.92 & 16396 & 6830 & 16726 & 0.033 \\ \hline

        \end{tabular}%
        }
        \caption{Performance metrics of passed and failed runs,including execution time, token consumption, and success rate in three stages. Model: phi4:14b, Date: 03.18}
        \label{tab:performance_metrics_ave_phi4:14b_03.18}
        \end{table}
    

    \begin{table}[!h]
        \centering
        \begin{tabular}{|l|r|r|r|r|}
        \hline
        \multicolumn{1}{|c|}{Stages} &
          \multicolumn{1}{c|}{Ave} &
          \multicolumn{1}{c|}{Med} &
          \multicolumn{1}{c|}{Max} &
          \multicolumn{1}{c|}{Min} \\ \hline
    DP & 47.97 & 43.81 & 98.57 & 32.06 \\ \hline
MC & 11.87 & 5.13 & 41.03 & 3.83 \\ \hline
SG & 97.60 & 89.81 & 260.94 & 30.36 \\ \hline

        \end{tabular}
        \caption{Time consumption (seconds) in three stages. Model: phi4:14b, Date: 03.18}
        \label{tab:time_phi4:14b_03.18}
        \end{table}
    
\newpage
\begin{figure}[!h]
    \centering
    \includegraphics[width=0.75\linewidth]{data_analysis/2025/03.18/tex/scatter_plot_phi4:14b_03.18.pdf}
    \caption{Distribution of time and token consumption in three tasks. Model: phi4:14b, Date: 03.18 (Date of DP and MC is 01.15).}
    \label{fig:scatter_phi4:14b_03.18}
\end{figure}

\begin{figure}[!h]
    \centering
    \includegraphics[width=0.75\linewidth]{data_analysis/2025/03.18/tex/bar_plot_phi4:14b_03.18.pdf}
    \caption{Average Time and Token Consumption by Task Status. Model: phi4:14b, Date: 03.18. \\ \textit{The average time (top) and token consumption (bottom) are compared across three stages, broken down by overall performance (considering all runs), the performance of only successful runs, and of only the failed runs.}}
    \label{fig:bar_phi4:14b_03.18}
\end{figure}



\clearpage 
\subsection{Results: 2025.03.18, Model: llama3.1:8b}
\textcolor{red}{Note: The results of DP and MC are from 2025.01.15, and the results of SG are from 2025.03.18.
}

    \begin{table}[!h]
        \centering
        \resizebox{\textwidth}{!}{%
        \begin{tabular}{|l|r|r|r|r|r|r|r|r|r|r|r|r|}
        \hline
        \multicolumn{1}{|c|}{} &
          \multicolumn{4}{c|}{Overall Token} &
          \multicolumn{4}{c|}{Input Token} &
          \multicolumn{4}{c|}{Output Token} \\ \cline{2-13}
        \multicolumn{1}{|c|}{\multirow{-2}{*}{Stages}} &
          \multicolumn{1}{c|}{Ave} &
          \multicolumn{1}{c|}{Med} &
          \multicolumn{1}{c|}{Max} &
          \multicolumn{1}{c|}{Min} &
          \multicolumn{1}{c|}{Ave} &
          \multicolumn{1}{c|}{Med} &
          \multicolumn{1}{c|}{Max} &
          \multicolumn{1}{c|}{Min} &
          \multicolumn{1}{c|}{Ave} &
          \multicolumn{1}{c|}{Med} &
          \multicolumn{1}{c|}{Max} &
          \multicolumn{1}{c|}{Min} \\ \hline
    DP & 16869 & 16563 & 35068 & 2076 & 13469 & 13143 & 26029 & 1782 & 3400 & 3420 & 9039 & 294 \\ \hline
MC & 4636 & 5805 & 8555 & 523 & 3797 & 4880 & 6960 & 389 & 838 & 908 & 1595 & 134 \\ \hline
SG & 14698 & 14776 & 16833 & 9929 & 11023 & 11058 & 12715 & 8095 & 3674 & 3761 & 4992 & 1834 \\ \hline

        \end{tabular}%
        }
        \caption{Token consumption details regarding Overall, Input, and Output in three stages. Model: llama3.1:8b, Date: 03.18}
        \label{tab:token_details_llama3.1:8b_03.18}
        \end{table}
    
\input{data_analysis/2025/03.18/tex/tab_performance_metrics_llama3.1:8b_03.18}
% \input{data_analysis/2025/03.18/tex/tab_cost_details_llama3.1:8b_03.18}

    \begin{table}[!h]
        \centering
        \resizebox{\textwidth}{!}{%
        \begin{tabular}{|l|r|r|r|r|r|r|r|}
        \hline
        \multicolumn{1}{|c|}{} &
          \multicolumn{3}{c|}{Ave Execution Time (s)} &
          \multicolumn{3}{c|}{Ave Total Tokens} &
          \multicolumn{1}{c|}{} \\ \cline{2-7}
        \multicolumn{1}{|c|}{\multirow{-2}{*}{Stages}} &
          \multicolumn{1}{c|}{Overall} &
          \multicolumn{1}{c|}{Pass} &
          \multicolumn{1}{c|}{Fail} &
          \multicolumn{1}{c|}{Overall} &
          \multicolumn{1}{c|}{Pass} &
          \multicolumn{1}{c|}{Fail} &
          \multicolumn{1}{|c|}{\multirow{-2}{*}{\shortstack{Success\\Rate}}} \\ \hline
    DP & 35.44 & 36.30 & 35.23 & 16869 & 17621 & 16681 & 0.200 \\ \hline
MC & 8.77 & 6.28 & 13.09 & 4636 & 3119 & 7256 & 0.633 \\ \hline
SG & 45.12 & 0.00 & 45.12 & 14698 & 0 & 14698 & 0.000 \\ \hline

        \end{tabular}%
        }
        \caption{Performance metrics of passed and failed runs,including execution time, token consumption, and success rate in three stages. Model: llama3.1:8b, Date: 03.18}
        \label{tab:performance_metrics_ave_llama3.1:8b_03.18}
        \end{table}
    

    \begin{table}[!h]
        \centering
        \begin{tabular}{|l|r|r|r|r|}
        \hline
        \multicolumn{1}{|c|}{Stages} &
          \multicolumn{1}{c|}{Ave} &
          \multicolumn{1}{c|}{Med} &
          \multicolumn{1}{c|}{Max} &
          \multicolumn{1}{c|}{Min} \\ \hline
    DP & 35.44 & 35.41 & 92.65 & 3.11 \\ \hline
MC & 8.77 & 9.88 & 16.66 & 1.28 \\ \hline
SG & 45.12 & 43.50 & 92.20 & 19.60 \\ \hline

        \end{tabular}
        \caption{Time consumption (seconds) in three stages. Model: llama3.1:8b, Date: 03.18}
        \label{tab:time_llama3.1:8b_03.18}
        \end{table}
    
\newpage
\begin{figure}[!h]
    \centering
    \includegraphics[width=0.75\linewidth]{data_analysis/2025/03.18/tex/scatter_plot_llama3.1:8b_03.18.pdf}
    \caption{Distribution of time and token consumption in three tasks. Model: llama3.1:8b, Date: 03.18 (Date of DP and MC is 01.15).}
    \label{fig:scatter_llama3.1:8b_03.18}
\end{figure}

\begin{figure}[!h]
    \centering
    \includegraphics[width=0.75\linewidth]{data_analysis/2025/03.18/tex/bar_plot_llama3.1:8b_03.18.pdf}
    \caption{Average Time and Token Consumption by Task Status. Model: llama3.1:8b, Date: 03.18. \\ \textit{The average time (top) and token consumption (bottom) are compared across three stages, broken down by overall performance (considering all runs), performance of only successful runs, and of only the failed runs.}}
    \label{fig:bar_llama3.1:8b_03.18}
\end{figure}



\clearpage 
\subsection{Results: 2025.03.18, Model: qwen2.5-coder:14b}
\textcolor{red}{Note: The results of DP and MC are from 2025.01.23, and the results of SG are from 2025.03.18.
}
\FloatBarrier
\input{data_analysis/2025/03.18/tex/tab_token_details_qwen2.5-coder:14b_03.18}

    \begin{table}[!h]
        \centering
        \resizebox{\textwidth}{!}{%
        \begin{tabular}{|l|r|r|r|r|r|r|r|}
        \hline
        \multicolumn{1}{|c|}{} &
          \multicolumn{3}{c|}{Execution Time (s)} &
          \multicolumn{3}{c|}{Token Consumption} &
          \multicolumn{1}{c|}{} \\ \cline{2-7}
        \multicolumn{1}{|c|}{\multirow{-2}{*}{Stages}} &
          \multicolumn{1}{c|}{Ave} &
          \multicolumn{1}{c|}{Max} &
          \multicolumn{1}{c|}{Min} &
          \multicolumn{1}{c|}{Ave} &
          \multicolumn{1}{c|}{Max} &
          \multicolumn{1}{c|}{Min} &
          \multicolumn{1}{|c|}{\multirow{-2}{*}{\shortstack{Success\\Rate}}} \\ \hline
    DP & 52.46 & 138.04 & 26.02 & 16929 & 27965 & 12076 & 0.300 \\ \hline
MC & 22.16 & 36.77 & 7.23 & 6106 & 9488 & 2420 & 0.567 \\ \hline
SG & 121.34 & 299.85 & 88.49 & 17485 & 18556 & 16050 & 0.000 \\ \hline

        \end{tabular}%
        }
        \caption{Performance metrics including execution time, token consumption, and success rate in three stages. Model: qwen2.5-coder:14b, Date: 03.18}
        \label{tab:performance_metrics_qwen2.5-coder:14b_03.18}
        \end{table}
    
% \input{data_analysis/2025/03.18/tex/tab_cost_details_qwen2.5-coder:14b_03.18}
\input{data_analysis/2025/03.18/tex/tab_performance_metrics_ave_qwen2.5-coder:14b_03.18}
\input{data_analysis/2025/03.18/tex/tab_time_qwen2.5-coder:14b_03.18}
\newpage
\begin{figure}[!h]
    \centering
    \includegraphics[width=0.75\linewidth]{data_analysis/2025/03.18/tex/scatter_plot_qwen2.5-coder:14b_03.18.pdf}
    \caption{Distribution of time and token consumption in three tasks. Model: qwen2.5-coder:14b, Date: 03.18.}
    \label{fig:scatter_qwen2.5-coder:14b_03.18}
\end{figure}

\begin{figure}[!h]
    \centering
    \includegraphics[width=0.75\linewidth]{data_analysis/2025/03.18/tex/bar_plot_qwen2.5-coder:14b_03.18.pdf}
    \caption{Average Time and Token Consumption by Task Status. Model: qwen2.5-coder:14b, Date: 03.18. \\ \textit{The average time (top) and token consumption (bottom) are compared across three stages, broken down by overall performance (considering all runs), the performance of only successful runs, and of only the failed runs.}}
    \label{fig:bar_qwen2.5-coder:14b_03.18}
\end{figure}



\clearpage 
\subsection{Results: 2025.03.18, Model: deepseek-r1:14b}
\textcolor{red}{Note: The results of DP and MC are from 2025.01.23, and the results of SG are from 2025.03.18.s}\\


In this test, we used Deepseek's \texttt{deepseek-r1:14b}, which is a reasoning LLM in their reasoning series of LLMs. Similar to OpenAI's \texttt{o1} series, it trades off speed for more powerful reasoning ability. It also outputs the reasoning process before the real answer to the question. A typical output is shown below \ref{fig:deepseekr1_example}. Although the model is powerful in its reasoning capability according to recent studies and tests, our test results based on using it's direct output as engineering code are not outstanding. Perhaps the reasoning model series can be excellent for providing helpful insights into human-in-the-loop engineering tasks instead.
\begin{figure}[!h]
    \centering
\includegraphics[width=0.75\linewidth]{data_analysis/01.23/tex/deepseekr1_example.png}
\caption{An example output of \texttt{deepseek-r1:14b}.}
\label{fig:deepseekr1_example}
\end{figure}

    \begin{table}[!h]
        \centering
        \resizebox{\textwidth}{!}{%
        \begin{tabular}{|l|r|r|r|r|r|r|r|r|r|r|r|r|}
        \hline
        \multicolumn{1}{|c|}{} &
          \multicolumn{4}{c|}{Overall Token} &
          \multicolumn{4}{c|}{Input Token} &
          \multicolumn{4}{c|}{Output Token} \\ \cline{2-13}
        \multicolumn{1}{|c|}{\multirow{-2}{*}{Stages}} &
          \multicolumn{1}{c|}{Ave} &
          \multicolumn{1}{c|}{Med} &
          \multicolumn{1}{c|}{Max} &
          \multicolumn{1}{c|}{Min} &
          \multicolumn{1}{c|}{Ave} &
          \multicolumn{1}{c|}{Med} &
          \multicolumn{1}{c|}{Max} &
          \multicolumn{1}{c|}{Min} &
          \multicolumn{1}{c|}{Ave} &
          \multicolumn{1}{c|}{Med} &
          \multicolumn{1}{c|}{Max} &
          \multicolumn{1}{c|}{Min} \\ \hline
    DP & 31304 & 29202 & 57981 & 13742 & 18742 & 17441 & 35044 & 7925 & 12562 & 11416 & 22937 & 5391 \\ \hline
MC & 8589 & 9334 & 11289 & 1248 & 4991 & 5462 & 6716 & 381 & 3598 & 3627 & 5213 & 867 \\ \hline
SG & 20260 & 20016 & 23179 & 18171 & 11358 & 11322 & 12891 & 11123 & 8901 & 8690 & 11841 & 6919 \\ \hline

        \end{tabular}%
        }
        \caption{Token consumption details regarding Overall, Input, and Output in three stages. Model: deepseek-r1:14b, Date: 03.18}
        \label{tab:token_details_deepseek-r1:14b_03.18}
        \end{table}
    
\input{data_analysis/2025/03.18/tex/tab_performance_metrics_deepseek-r1:14b_03.18}
% \input{data_analysis/2025/03.18/tex/tab_cost_details_deepseek-r1:14b_03.18}

    \begin{table}[!h]
        \centering
        \resizebox{\textwidth}{!}{%
        \begin{tabular}{|l|r|r|r|r|r|r|r|}
        \hline
        \multicolumn{1}{|c|}{} &
          \multicolumn{3}{c|}{Ave Execution Time (s)} &
          \multicolumn{3}{c|}{Ave Total Tokens} &
          \multicolumn{1}{c|}{} \\ \cline{2-7}
        \multicolumn{1}{|c|}{\multirow{-2}{*}{Stages}} &
          \multicolumn{1}{c|}{Overall} &
          \multicolumn{1}{c|}{Pass} &
          \multicolumn{1}{c|}{Fail} &
          \multicolumn{1}{c|}{Overall} &
          \multicolumn{1}{c|}{Pass} &
          \multicolumn{1}{c|}{Fail} &
          \multicolumn{1}{|c|}{\multirow{-2}{*}{\shortstack{Success\\Rate}}} \\ \hline
    DP & 204.56 & 274.67 & 179.07 & 31304 & 40695 & 27890 & 0.267 \\ \hline
MC & 58.05 & 46.12 & 64.02 & 8589 & 6435 & 9667 & 0.333 \\ \hline
SG & 201.29 & 0.00 & 201.29 & 20260 & 0 & 20260 & 0.000 \\ \hline

        \end{tabular}%
        }
        \caption{Performance metrics of passed and failed runs,including execution time, token consumption, and success rate in three stages. Model: deepseek-r1:14b, Date: 03.18}
        \label{tab:performance_metrics_ave_deepseek-r1:14b_03.18}
        \end{table}
    

    \begin{table}[!h]
        \centering
        \begin{tabular}{|l|r|r|r|r|}
        \hline
        \multicolumn{1}{|c|}{Stages} &
          \multicolumn{1}{c|}{Ave} &
          \multicolumn{1}{c|}{Med} &
          \multicolumn{1}{c|}{Max} &
          \multicolumn{1}{c|}{Min} \\ \hline
    DP & 204.56 & 184.48 & 373.34 & 87.83 \\ \hline
MC & 58.05 & 58.94 & 84.15 & 13.24 \\ \hline
SG & 201.29 & 179.37 & 557.00 & 126.79 \\ \hline

        \end{tabular}
        \caption{Time consumption (seconds) in three stages. Model: deepseek-r1:14b, Date: 03.18}
        \label{tab:time_deepseek-r1:14b_03.18}
        \end{table}
    
\newpage
\begin{figure}[!h]
    \centering
    \includegraphics[width=0.75\linewidth]{data_analysis/2025/03.18/tex/scatter_plot_deepseek-r1:14b_03.18.pdf}
    \caption{Distribution of time and token consumption in three tasks. Model: deepseek-r1:14b, Date: 03.18.}
    \label{fig:scatter_deepseek-r1:14b_03.18}
\end{figure}

\begin{figure}[!h]
    \centering
    \includegraphics[width=0.75\linewidth]{data_analysis/2025/03.18/tex/bar_plot_deepseek-r1:14b_03.18.pdf}
    \caption{Average Time and Token Consumption by Task Status. Model: deepseek-r1:14b, Date: 03.18. \\ \textit{The average time (top) and token consumption (bottom) are compared across three stages, broken down by overall performance (considering all runs), performance of only successful runs, and of only the failed runs.}}
    \label{fig:bar_deepseek-r1:14b_03.18}
\end{figure}


\clearpage 
\newpage
\subsection{Results: 2025.03.22, Model: codestral:22b}
 

\FloatBarrier

    \begin{table}[!h]
        \centering
        \resizebox{\textwidth}{!}{%
        \begin{tabular}{|l|r|r|r|r|r|r|r|r|r|r|r|r|}
        \hline
        \multicolumn{1}{|c|}{} &
          \multicolumn{4}{c|}{Overall Token} &
          \multicolumn{4}{c|}{Input Token} &
          \multicolumn{4}{c|}{Output Token} \\ \cline{2-13}
        \multicolumn{1}{|c|}{\multirow{-2}{*}{Stages}} &
          \multicolumn{1}{c|}{Ave} &
          \multicolumn{1}{c|}{Med} &
          \multicolumn{1}{c|}{Max} &
          \multicolumn{1}{c|}{Min} &
          \multicolumn{1}{c|}{Ave} &
          \multicolumn{1}{c|}{Med} &
          \multicolumn{1}{c|}{Max} &
          \multicolumn{1}{c|}{Min} &
          \multicolumn{1}{c|}{Ave} &
          \multicolumn{1}{c|}{Med} &
          \multicolumn{1}{c|}{Max} &
          \multicolumn{1}{c|}{Min} \\ \hline
    DP & 22783 & 22377 & 35413 & 13801 & 18363 & 17754 & 29326 & 11267 & 4420 & 4457 & 6415 & 2534 \\ \hline
MC & 7561 & 8169 & 10071 & 1027 & 5879 & 6283 & 8031 & 484 & 1681 & 1705 & 2326 & 543 \\ \hline
SG & 18509 & 18629 & 24046 & 9753 & 14091 & 14011 & 18107 & 7864 & 4417 & 4221 & 5939 & 1889 \\ \hline

        \end{tabular}%
        }
        \caption{Token consumption details regarding Overall, Input, and Output in three stages. Model: codestral:latest, Date: 03.22}
        \label{tab:token_details_codestral:latest_03.22}
        \end{table}
    

    \begin{table}[!h]
        \centering
        \resizebox{\textwidth}{!}{%
        \begin{tabular}{|l|r|r|r|r|r|r|r|}
        \hline
        \multicolumn{1}{|c|}{} &
          \multicolumn{3}{c|}{Execution Time (s)} &
          \multicolumn{3}{c|}{Token Consumption} &
          \multicolumn{1}{c|}{} \\ \cline{2-7}
        \multicolumn{1}{|c|}{\multirow{-2}{*}{Stages}} &
          \multicolumn{1}{c|}{Ave} &
          \multicolumn{1}{c|}{Max} &
          \multicolumn{1}{c|}{Min} &
          \multicolumn{1}{c|}{Ave} &
          \multicolumn{1}{c|}{Max} &
          \multicolumn{1}{c|}{Min} &
          \multicolumn{1}{|c|}{\multirow{-2}{*}{\shortstack{Success\\Rate}}} \\ \hline
    DP & 102.46 & 153.41 & 61.70 & 22783 & 35413 & 13801 & 0.267 \\ \hline
MC & 37.70 & 54.64 & 9.77 & 7561 & 10071 & 1027 & 0.400 \\ \hline
SG & 102.00 & 148.94 & 46.18 & 18509 & 24046 & 9753 & 0.118 \\ \hline

        \end{tabular}%
        }
        \caption{Performance metrics including execution time, token consumption, and success rate in three stages. Model: codestral:latest, Date: 03.22}
        \label{tab:performance_metrics_codestral:latest_03.22}
        \end{table}
    
% \input{data_analysis/2025/03.22/tex/tab_cost_details_codestral:latest_03.22}

    \begin{table}[!h]
        \centering
        \resizebox{\textwidth}{!}{%
        \begin{tabular}{|l|r|r|r|r|r|r|r|}
        \hline
        \multicolumn{1}{|c|}{} &
          \multicolumn{3}{c|}{Ave Execution Time (s)} &
          \multicolumn{3}{c|}{Ave Total Tokens} &
          \multicolumn{1}{c|}{} \\ \cline{2-7}
        \multicolumn{1}{|c|}{\multirow{-2}{*}{Stages}} &
          \multicolumn{1}{c|}{Overall} &
          \multicolumn{1}{c|}{Pass} &
          \multicolumn{1}{c|}{Fail} &
          \multicolumn{1}{c|}{Overall} &
          \multicolumn{1}{c|}{Pass} &
          \multicolumn{1}{c|}{Fail} &
          \multicolumn{1}{|c|}{\multirow{-2}{*}{\shortstack{Success\\Rate}}} \\ \hline
    DP & 102.46 & 93.61 & 105.67 & 22783 & 20783 & 23511 & 0.267 \\ \hline
MC & 37.70 & 31.75 & 41.67 & 7561 & 5952 & 8633 & 0.400 \\ \hline
SG & 102.00 & 58.28 & 107.83 & 18509 & 11327 & 19466 & 0.118 \\ \hline

        \end{tabular}%
        }
        \caption{Performance metrics of passed and failed runs,including execution time, token consumption, and success rate in three stages. Model: codestral:latest, Date: 03.22}
        \label{tab:performance_metrics_ave_codestral:latest_03.22}
        \end{table}
    

    \begin{table}[!h]
        \centering
        \begin{tabular}{|l|r|r|r|r|}
        \hline
        \multicolumn{1}{|c|}{Stages} &
          \multicolumn{1}{c|}{Ave} &
          \multicolumn{1}{c|}{Med} &
          \multicolumn{1}{c|}{Max} &
          \multicolumn{1}{c|}{Min} \\ \hline
    DP & 102.46 & 102.87 & 153.41 & 61.70 \\ \hline
MC & 37.70 & 38.95 & 54.64 & 9.77 \\ \hline
SG & 102.00 & 108.04 & 148.94 & 46.18 \\ \hline

        \end{tabular}
        \caption{Time consumption (seconds) in three stages. Model: codestral:latest, Date: 03.22}
        \label{tab:time_codestral:latest_03.22}
        \end{table}
    
\newpage
\begin{figure}[!h]
    \centering
    \includegraphics[width=0.75\linewidth]{data_analysis/2025/03.22/tex/scatter_plot_codestral:latest_03.22.pdf}
    \caption{Distribution of time and token consumption in three tasks. Model: codestral:22b, Date: 03.22 (Date of DP and MC is 01.15).}
    \label{fig:scatter_codestral:latest_03.22}
\end{figure}

\begin{figure}[!h]
    \centering
    \includegraphics[width=0.75\linewidth]{data_analysis/2025/03.22/tex/bar_plot_codestral:latest_03.22.pdf}
    \caption{Average Time and Token Consumption by Task Status. Model: codestral:22b, Date: 03.22. \\ \textit{The average time (top) and token consumption (bottom) are compared across three stages, broken down by overall performance (considering all runs), the performance of only successful runs, and of only the failed runs.}}
    \label{fig:bar_codestral:latest_03.22}
\end{figure}

 

\clearpage 
\input{data_analysis/2025/03.27/tex/03.27}
\clearpage 
\input{data_analysis/2025/03.28/tex/03.28}
\clearpage 
\newpage
\subsection{05.19, qwen2.5-coder:32b, Params: True}

\subsubsection*{\texttt{PySG only, batch\_uid:6eca}}
\addcontentsline{toc}{subsubsection}{\texttt{PySG only, batch\_uid:6eca}}

\FloatBarrier
\input{data_analysis/2025/05.19/tex/tab_cost_details_qwen2.5-coder:32b_05.19_6eca.tex}

    \begin{table}[!h]
        \centering
        \resizebox{\textwidth}{!}{%
        \begin{tabular}{|l|r|r|r|r|r|r|r|}
        \hline
        \multicolumn{1}{|c|}{} &
          \multicolumn{3}{c|}{Ave Execution Time (s)} &
          \multicolumn{3}{c|}{Ave Total Tokens} &
          \multicolumn{1}{c|}{} \\ \cline{2-7}
        \multicolumn{1}{|c|}{\multirow{-2}{*}{Stages}} &
          \multicolumn{1}{c|}{Overall} &
          \multicolumn{1}{c|}{Pass} &
          \multicolumn{1}{c|}{Fail} &
          \multicolumn{1}{c|}{Overall} &
          \multicolumn{1}{c|}{Pass} &
          \multicolumn{1}{c|}{Fail} &
          \multicolumn{1}{|c|}{\multirow{-2}{*}{\shortstack{Success\\Rate}}} \\ \hline
    PSG & 46.37 & 46.37 & 0.00 & 2513 & 2513 & 0 & 1.000 \\ \hline

        \end{tabular}%
        }
        \caption{Performance metrics of passed and failed runs,including execution time, token consumption, and success rate in three stages. Batch ID: 6eca,Model: qwen2.5-coder:32b, Date: 05.19, Parameters: temperature=0.1, top\_p=0.3}
        \label{tab:performance_metrics_ave_6eca_qwen2.5-coder:32b_05.19}
        \end{table}
    

    \begin{table}[!h]
        \centering
        \resizebox{\textwidth}{!}{%
        \begin{tabular}{|l|r|r|r|r|r|r|r|}
        \hline
        \multicolumn{1}{|c|}{} &
          \multicolumn{3}{c|}{Execution Time (s)} &
          \multicolumn{3}{c|}{Token Consumption} &
          \multicolumn{1}{c|}{} \\ \cline{2-7}
        \multicolumn{1}{|c|}{\multirow{-2}{*}{Stages}} &
          \multicolumn{1}{c|}{Ave} &
          \multicolumn{1}{c|}{Max} &
          \multicolumn{1}{c|}{Min} &
          \multicolumn{1}{c|}{Ave} &
          \multicolumn{1}{c|}{Max} &
          \multicolumn{1}{c|}{Min} &
          \multicolumn{1}{|c|}{\multirow{-2}{*}{\shortstack{Success\\Rate}}} \\ \hline
    PSG & 46.37 & 418.85 & 26.35 & 2513 & 2533 & 2503 & 1.000 \\ \hline

        \end{tabular}%
        }
        \caption{Performance metrics including execution time, token consumption, and success rate in three stages. Batch ID: 6eca,Model: qwen2.5-coder:32b, Date: 05.19, Parameters: temperature=0.1, top\_p=0.3}
        \label{tab:performance_metrics_6eca_qwen2.5-coder:32b_05.19}
        \end{table}
    

    \begin{table}[!h]
        \centering
        \begin{tabular}{|l|r|r|r|r|}
        \hline
        \multicolumn{1}{|c|}{Stages} &
          \multicolumn{1}{c|}{Ave} &
          \multicolumn{1}{c|}{Med} &
          \multicolumn{1}{c|}{Max} &
          \multicolumn{1}{c|}{Min} \\ \hline
    PSG & 46.37 & 26.81 & 418.85 & 26.35 \\ \hline

        \end{tabular}
        \caption{Time consumption (seconds) in three stages. Batch ID: 6eca,Model: qwen2.5-coder:32b, Date: 05.19, Parameters: temperature=0.1, top\_p=0.3}
        \label{tab:time_6eca_qwen2.5-coder:32b_05.19}
        \end{table}
    

    \begin{table}[!h]
        \centering
        \resizebox{\textwidth}{!}{%
        \begin{tabular}{|l|r|r|r|r|r|r|r|r|r|r|r|r|}
        \hline
        \multicolumn{1}{|c|}{} &
          \multicolumn{4}{c|}{Overall Token} &
          \multicolumn{4}{c|}{Input Token} &
          \multicolumn{4}{c|}{Output Token} \\ \cline{2-13}
        \multicolumn{1}{|c|}{\multirow{-2}{*}{Stages}} &
          \multicolumn{1}{c|}{Ave} &
          \multicolumn{1}{c|}{Med} &
          \multicolumn{1}{c|}{Max} &
          \multicolumn{1}{c|}{Min} &
          \multicolumn{1}{c|}{Ave} &
          \multicolumn{1}{c|}{Med} &
          \multicolumn{1}{c|}{Max} &
          \multicolumn{1}{c|}{Min} &
          \multicolumn{1}{c|}{Ave} &
          \multicolumn{1}{c|}{Med} &
          \multicolumn{1}{c|}{Max} &
          \multicolumn{1}{c|}{Min} \\ \hline
    PSG & 2513 & 2515 & 2533 & 2503 & 1580 & 1580 & 1580 & 1580 & 933 & 935 & 953 & 923 \\ \hline

        \end{tabular}%
        }
        \caption{Token consumption details regarding Overall, Input, and Output in three stages. Batch ID: 6eca, Model: qwen2.5-coder:32b, Date: 05.19, Parameters: temperature=0.1, top\_p=0.3}
        \label{tab:token_details_6eca_qwen2.5-coder:32b_05.19}
        \end{table}
    
\newpage
\begin{figure}[!h]
    \centering
    \includegraphics[width=0.75\linewidth]{data_analysis/2025/05.19/tex/bar_plot_qwen2.5-coder:32b_05.19_6eca.pdf}
    \caption{Average Time and Token Consumption by Task Status. Model: qwen2.5-coder:32b, Date: 05.19, Parameters: temperature=0.1, top\_p=0.3 \\ \textit{The average time (top) and token consumption (bottom) are compared across three stages, broken down by overall performance (considering all runs), performance of only successful runs, and of only the failed runs.}}
    \label{fig:bar_plot_qwen2.5-coder:32b_05.19_6eca}
\end{figure}

\begin{figure}[!h]
    \centering
    \includegraphics[width=0.75\linewidth]{data_analysis/2025/05.19/tex/scatter_plot_qwen2.5-coder:32b_05.19_6eca.pdf}
    \caption{Scatter Plot Qwen2.5-Coder:32B 05.19 6EcaDistribution of time and token consumption in three tasks. Model: qwen2.5-coder:32b, Date: 05.19, Parameters: temperature=0.1, top\_p=0.3}
    \label{fig:scatter_plot_qwen2.5-coder:32b_05.19_6eca}
\end{figure}

\newpage
\subsection{05.19, qwen2.5-coder:32b, Params: False}

\subsubsection*{\texttt{PySG only, batch\_uid:aabd}}
\addcontentsline{toc}{subsubsection}{\texttt{PySG only, batch\_uid:aabd}}

\FloatBarrier
\input{data_analysis/2025/05.19/tex/tab_cost_details_qwen2.5-coder:32b_05.19_aabd.tex}

    \begin{table}[!h]
        \centering
        \resizebox{\textwidth}{!}{%
        \begin{tabular}{|l|r|r|r|r|r|r|r|}
        \hline
        \multicolumn{1}{|c|}{} &
          \multicolumn{3}{c|}{Ave Execution Time (s)} &
          \multicolumn{3}{c|}{Ave Total Tokens} &
          \multicolumn{1}{c|}{} \\ \cline{2-7}
        \multicolumn{1}{|c|}{\multirow{-2}{*}{Stages}} &
          \multicolumn{1}{c|}{Overall} &
          \multicolumn{1}{c|}{Pass} &
          \multicolumn{1}{c|}{Fail} &
          \multicolumn{1}{c|}{Overall} &
          \multicolumn{1}{c|}{Pass} &
          \multicolumn{1}{c|}{Fail} &
          \multicolumn{1}{|c|}{\multirow{-2}{*}{\shortstack{Success\\Rate}}} \\ \hline
    PSG & 83.67 & 39.91 & 204.03 & 5907 & 2790 & 14479 & 0.733 \\ \hline

        \end{tabular}%
        }
        \caption{Performance metrics of passed and failed runs,including execution time, token consumption, and success rate in three stages. Batch ID: aabd,Model: qwen2.5-coder:32b, Date: 05.19, Parameters: None}
        \label{tab:performance_metrics_ave_aabd_qwen2.5-coder:32b_05.19}
        \end{table}
    

    \begin{table}[!h]
        \centering
        \resizebox{\textwidth}{!}{%
        \begin{tabular}{|l|r|r|r|r|r|r|r|}
        \hline
        \multicolumn{1}{|c|}{} &
          \multicolumn{3}{c|}{Execution Time (s)} &
          \multicolumn{3}{c|}{Token Consumption} &
          \multicolumn{1}{c|}{} \\ \cline{2-7}
        \multicolumn{1}{|c|}{\multirow{-2}{*}{Stages}} &
          \multicolumn{1}{c|}{Ave} &
          \multicolumn{1}{c|}{Max} &
          \multicolumn{1}{c|}{Min} &
          \multicolumn{1}{c|}{Ave} &
          \multicolumn{1}{c|}{Max} &
          \multicolumn{1}{c|}{Min} &
          \multicolumn{1}{|c|}{\multirow{-2}{*}{\shortstack{Success\\Rate}}} \\ \hline
    PSG & 83.67 & 215.41 & 30.70 & 5907 & 14860 & 2291 & 0.733 \\ \hline

        \end{tabular}%
        }
        \caption{Performance metrics including execution time, token consumption, and success rate in three stages. Batch ID: aabd,Model: qwen2.5-coder:32b, Date: 05.19, Parameters: None}
        \label{tab:performance_metrics_aabd_qwen2.5-coder:32b_05.19}
        \end{table}
    

    \begin{table}[!h]
        \centering
        \begin{tabular}{|l|r|r|r|r|}
        \hline
        \multicolumn{1}{|c|}{Stages} &
          \multicolumn{1}{c|}{Ave} &
          \multicolumn{1}{c|}{Med} &
          \multicolumn{1}{c|}{Max} &
          \multicolumn{1}{c|}{Min} \\ \hline
    PSG & 83.67 & 36.83 & 215.41 & 30.70 \\ \hline

        \end{tabular}
        \caption{Time consumption (seconds) in three stages. Batch ID: aabd,Model: qwen2.5-coder:32b, Date: 05.19, Parameters: None}
        \label{tab:time_aabd_qwen2.5-coder:32b_05.19}
        \end{table}
    

    \begin{table}[!h]
        \centering
        \resizebox{\textwidth}{!}{%
        \begin{tabular}{|l|r|r|r|r|r|r|r|r|r|r|r|r|}
        \hline
        \multicolumn{1}{|c|}{} &
          \multicolumn{4}{c|}{Overall Token} &
          \multicolumn{4}{c|}{Input Token} &
          \multicolumn{4}{c|}{Output Token} \\ \cline{2-13}
        \multicolumn{1}{|c|}{\multirow{-2}{*}{Stages}} &
          \multicolumn{1}{c|}{Ave} &
          \multicolumn{1}{c|}{Med} &
          \multicolumn{1}{c|}{Max} &
          \multicolumn{1}{c|}{Min} &
          \multicolumn{1}{c|}{Ave} &
          \multicolumn{1}{c|}{Med} &
          \multicolumn{1}{c|}{Max} &
          \multicolumn{1}{c|}{Min} &
          \multicolumn{1}{c|}{Ave} &
          \multicolumn{1}{c|}{Med} &
          \multicolumn{1}{c|}{Max} &
          \multicolumn{1}{c|}{Min} \\ \hline
    PSG & 5907 & 2436 & 14860 & 2291 & 3964 & 1580 & 9772 & 1580 & 1943 & 856 & 5088 & 711 \\ \hline

        \end{tabular}%
        }
        \caption{Token consumption details regarding Overall, Input, and Output in three stages. Batch ID: aabd, Model: qwen2.5-coder:32b, Date: 05.19, Parameters: None}
        \label{tab:token_details_aabd_qwen2.5-coder:32b_05.19}
        \end{table}
    
\newpage
\begin{figure}[!h]
    \centering
    \includegraphics[width=0.75\linewidth]{data_analysis/2025/05.19/tex/bar_plot_qwen2.5-coder:32b_05.19_aabd.pdf}
    \caption{Average Time and Token Consumption by Task Status. Model: qwen2.5-coder:32b, Date: 05.19, Parameters: None \\ \textit{The average time (top) and token consumption (bottom) are compared across three stages, broken down by overall performance (considering all runs), performance of only successful runs, and of only the failed runs.}}
    \label{fig:bar_plot_qwen2.5-coder:32b_05.19_aabd}
\end{figure}

\begin{figure}[!h]
    \centering
    \includegraphics[width=0.75\linewidth]{data_analysis/2025/05.19/tex/scatter_plot_qwen2.5-coder:32b_05.19_aabd.pdf}
    \caption{Scatter Plot Qwen2.5-Coder:32B 05.19 AabdDistribution of time and token consumption in three tasks. Model: qwen2.5-coder:32b, Date: 05.19, Parameters: None}
    \label{fig:scatter_plot_qwen2.5-coder:32b_05.19_aabd}
\end{figure}


\clearpage 
\newpage
\subsection{05.21, codestral:latest, Params: None}

\subsubsection*{\texttt{PySG only, batch\_uid:366b}}

\FloatBarrier

    \begin{table}[!h]
        \centering
        \resizebox{\textwidth}{!}{%
        \begin{tabular}{|l|r|r|r|r|r|r|r|r|r|r|r|r|}
        \hline
        \multicolumn{1}{|c|}{} &
          \multicolumn{4}{c|}{Overall Cost (USD cents)} &
          \multicolumn{4}{c|}{Input Cost (USD cents)} &
          \multicolumn{4}{c|}{Output Cost (USD cents)} \\ \cline{2-13}
        \multicolumn{1}{|c|}{\multirow{-2}{*}{Stages}} &
          \multicolumn{1}{c|}{Ave} &
          \multicolumn{1}{c|}{Med} &
          \multicolumn{1}{c|}{Max} &
          \multicolumn{1}{c|}{Min} &
          \multicolumn{1}{c|}{Ave} &
          \multicolumn{1}{c|}{Med} &
          \multicolumn{1}{c|}{Max} &
          \multicolumn{1}{c|}{Min} &
          \multicolumn{1}{c|}{Ave} &
          \multicolumn{1}{c|}{Med} &
          \multicolumn{1}{c|}{Max} &
          \multicolumn{1}{c|}{Min} \\ \hline
    PSG & 0.00 & 0.00 & 0.00 & 0.00 & 0.00 & 0.00 & 0.00 & 0.00 & 0.00 & 0.00 & 0.00 & 0.00 \\ \hline

        \end{tabular}%
        }
        \caption{Cost details regarding Overall, Input, and Output in three stages.  Batch ID: 366b,Model: codestral:latest, Date: 05.21, Parameters: None}
        \label{tab:cost_details_366b_codestral:latest_05.21}
        \end{table}
    

    \begin{table}[!h]
        \centering
        \resizebox{\textwidth}{!}{%
        \begin{tabular}{|l|r|r|r|r|r|r|r|}
        \hline
        \multicolumn{1}{|c|}{} &
          \multicolumn{3}{c|}{Ave Execution Time (s)} &
          \multicolumn{3}{c|}{Ave Total Tokens} &
          \multicolumn{1}{c|}{} \\ \cline{2-7}
        \multicolumn{1}{|c|}{\multirow{-2}{*}{Stages}} &
          \multicolumn{1}{c|}{Overall} &
          \multicolumn{1}{c|}{Pass} &
          \multicolumn{1}{c|}{Fail} &
          \multicolumn{1}{c|}{Overall} &
          \multicolumn{1}{c|}{Pass} &
          \multicolumn{1}{c|}{Fail} &
          \multicolumn{1}{|c|}{\multirow{-2}{*}{\shortstack{Success\\Rate}}} \\ \hline
    PSG & 64.49 & 26.25 & 78.40 & 11268 & 4155 & 13855 & 0.267 \\ \hline

        \end{tabular}%
        }
        \caption{Performance metrics of passed and failed runs,including execution time, token consumption, and success rate in three stages. Batch ID: 366b,Model: codestral:latest, Date: 05.21, Parameters: None}
        \label{tab:performance_metrics_ave_366b_codestral:latest_05.21}
        \end{table}
    

    \begin{table}[!h]
        \centering
        \resizebox{\textwidth}{!}{%
        \begin{tabular}{|l|r|r|r|r|r|r|r|}
        \hline
        \multicolumn{1}{|c|}{} &
          \multicolumn{3}{c|}{Execution Time (s)} &
          \multicolumn{3}{c|}{Token Consumption} &
          \multicolumn{1}{c|}{} \\ \cline{2-7}
        \multicolumn{1}{|c|}{\multirow{-2}{*}{Stages}} &
          \multicolumn{1}{c|}{Ave} &
          \multicolumn{1}{c|}{Max} &
          \multicolumn{1}{c|}{Min} &
          \multicolumn{1}{c|}{Ave} &
          \multicolumn{1}{c|}{Max} &
          \multicolumn{1}{c|}{Min} &
          \multicolumn{1}{|c|}{\multirow{-2}{*}{\shortstack{Success\\Rate}}} \\ \hline
    PSG & 64.49 & 120.98 & 16.17 & 11268 & 15053 & 2645 & 0.267 \\ \hline

        \end{tabular}%
        }
        \caption{Performance metrics including execution time, token consumption, and success rate in three stages. Batch ID: 366b,Model: codestral:latest, Date: 05.21, Parameters: None}
        \label{tab:performance_metrics_366b_codestral:latest_05.21}
        \end{table}
    

    \begin{table}[!h]
        \centering
        \begin{tabular}{|l|r|r|r|r|}
        \hline
        \multicolumn{1}{|c|}{Stages} &
          \multicolumn{1}{c|}{Ave} &
          \multicolumn{1}{c|}{Med} &
          \multicolumn{1}{c|}{Max} &
          \multicolumn{1}{c|}{Min} \\ \hline
    PSG & 64.49 & 72.35 & 120.98 & 16.17 \\ \hline

        \end{tabular}
        \caption{Time consumption (seconds) in three stages. Batch ID: 366b,Model: codestral:latest, Date: 05.21, Parameters: None}
        \label{tab:time_366b_codestral:latest_05.21}
        \end{table}
    

    \begin{table}[!h]
        \centering
        \resizebox{\textwidth}{!}{%
        \begin{tabular}{|l|r|r|r|r|r|r|r|r|r|r|r|r|}
        \hline
        \multicolumn{1}{|c|}{} &
          \multicolumn{4}{c|}{Overall Token} &
          \multicolumn{4}{c|}{Input Token} &
          \multicolumn{4}{c|}{Output Token} \\ \cline{2-13}
        \multicolumn{1}{|c|}{\multirow{-2}{*}{Stages}} &
          \multicolumn{1}{c|}{Ave} &
          \multicolumn{1}{c|}{Med} &
          \multicolumn{1}{c|}{Max} &
          \multicolumn{1}{c|}{Min} &
          \multicolumn{1}{c|}{Ave} &
          \multicolumn{1}{c|}{Med} &
          \multicolumn{1}{c|}{Max} &
          \multicolumn{1}{c|}{Min} &
          \multicolumn{1}{c|}{Ave} &
          \multicolumn{1}{c|}{Med} &
          \multicolumn{1}{c|}{Max} &
          \multicolumn{1}{c|}{Min} \\ \hline
    PSG & 11268 & 13558 & 15053 & 2645 & 8080 & 10014 & 10014 & 1822 & 3188 & 3611 & 5039 & 823 \\ \hline

        \end{tabular}%
        }
        \caption{Token consumption details regarding Overall, Input, and Output in three stages. Batch ID: 366b, Model: codestral:latest, Date: 05.21, Parameters: None}
        \label{tab:token_details_366b_codestral:latest_05.21}
        \end{table}
    
\newpage
\begin{figure}[!h]
    \centering
    \includegraphics[width=0.75\linewidth]{data_analysis/2025/05.21/tex/bar_plot_codestral:latest_05.21_366b.pdf}
    \caption{Average Time and Token Consumption by Task Status. Model: codestral:latest, Date: 05.21, Parameters: None \\ \textit{The average time (top) and token consumption (bottom) are compared across three stages, broken down by overall performance (considering all runs), performance of only successful runs, and of only the failed runs.}}
    \label{fig:bar_plot_codestral:latest_05.21_366b}
\end{figure}

\begin{figure}[!h]
    \centering
    \includegraphics[width=0.75\linewidth]{data_analysis/2025/05.21/tex/scatter_plot_codestral:latest_05.21_366b.pdf}
    \caption{Scatter Plot Codestral:Latest 05.21 366BDistribution of time and token consumption in three tasks. Model: codestral:latest, Date: 05.21, Parameters: None}
    \label{fig:scatter_plot_codestral:latest_05.21_366b}
\end{figure}

\newpage
\subsection{05.21, codestral:latest, Params: temperature=0.1, top\_p=0.3}

\subsubsection*{\texttt{PySG only, batch\_uid:4b06}}

\FloatBarrier
\input{data_analysis/2025/05.21/tex/tab_cost_details_codestral:latest_05.21_4b06.tex}

    \begin{table}[!h]
        \centering
        \resizebox{\textwidth}{!}{%
        \begin{tabular}{|l|r|r|r|r|r|r|r|}
        \hline
        \multicolumn{1}{|c|}{} &
          \multicolumn{3}{c|}{Ave Execution Time (s)} &
          \multicolumn{3}{c|}{Ave Total Tokens} &
          \multicolumn{1}{c|}{} \\ \cline{2-7}
        \multicolumn{1}{|c|}{\multirow{-2}{*}{Stages}} &
          \multicolumn{1}{c|}{Overall} &
          \multicolumn{1}{c|}{Pass} &
          \multicolumn{1}{c|}{Fail} &
          \multicolumn{1}{c|}{Overall} &
          \multicolumn{1}{c|}{Pass} &
          \multicolumn{1}{c|}{Fail} &
          \multicolumn{1}{|c|}{\multirow{-2}{*}{\shortstack{Success\\Rate}}} \\ \hline
    PSG & 76.87 & 0.00 & 76.87 & 13853 & 0 & 13853 & 0.000 \\ \hline

        \end{tabular}%
        }
        \caption{Performance metrics of passed and failed runs,including execution time, token consumption, and success rate in three stages. Batch ID: 4b06,Model: codestral:latest, Date: 05.21, Parameters: temperature=0.1, top\_p=0.3}
        \label{tab:performance_metrics_ave_4b06_codestral:latest_05.21}
        \end{table}
    

    \begin{table}[!h]
        \centering
        \resizebox{\textwidth}{!}{%
        \begin{tabular}{|l|r|r|r|r|r|r|r|}
        \hline
        \multicolumn{1}{|c|}{} &
          \multicolumn{3}{c|}{Execution Time (s)} &
          \multicolumn{3}{c|}{Token Consumption} &
          \multicolumn{1}{c|}{} \\ \cline{2-7}
        \multicolumn{1}{|c|}{\multirow{-2}{*}{Stages}} &
          \multicolumn{1}{c|}{Ave} &
          \multicolumn{1}{c|}{Max} &
          \multicolumn{1}{c|}{Min} &
          \multicolumn{1}{c|}{Ave} &
          \multicolumn{1}{c|}{Max} &
          \multicolumn{1}{c|}{Min} &
          \multicolumn{1}{|c|}{\multirow{-2}{*}{\shortstack{Success\\Rate}}} \\ \hline
    PSG & 76.87 & 95.37 & 60.69 & 13853 & 14707 & 13007 & 0.000 \\ \hline

        \end{tabular}%
        }
        \caption{Performance metrics including execution time, token consumption, and success rate in three stages. Batch ID: 4b06,Model: codestral:latest, Date: 05.21, Parameters: temperature=0.1, top\_p=0.3}
        \label{tab:performance_metrics_4b06_codestral:latest_05.21}
        \end{table}
    

    \begin{table}[!h]
        \centering
        \begin{tabular}{|l|r|r|r|r|}
        \hline
        \multicolumn{1}{|c|}{Stages} &
          \multicolumn{1}{c|}{Ave} &
          \multicolumn{1}{c|}{Med} &
          \multicolumn{1}{c|}{Max} &
          \multicolumn{1}{c|}{Min} \\ \hline
    PSG & 76.87 & 76.42 & 95.37 & 60.69 \\ \hline

        \end{tabular}
        \caption{Time consumption (seconds) in three stages. Batch ID: 4b06,Model: codestral:latest, Date: 05.21, Parameters: temperature=0.1, top\_p=0.3}
        \label{tab:time_4b06_codestral:latest_05.21}
        \end{table}
    

    \begin{table}[!h]
        \centering
        \resizebox{\textwidth}{!}{%
        \begin{tabular}{|l|r|r|r|r|r|r|r|r|r|r|r|r|}
        \hline
        \multicolumn{1}{|c|}{} &
          \multicolumn{4}{c|}{Overall Token} &
          \multicolumn{4}{c|}{Input Token} &
          \multicolumn{4}{c|}{Output Token} \\ \cline{2-13}
        \multicolumn{1}{|c|}{\multirow{-2}{*}{Stages}} &
          \multicolumn{1}{c|}{Ave} &
          \multicolumn{1}{c|}{Med} &
          \multicolumn{1}{c|}{Max} &
          \multicolumn{1}{c|}{Min} &
          \multicolumn{1}{c|}{Ave} &
          \multicolumn{1}{c|}{Med} &
          \multicolumn{1}{c|}{Max} &
          \multicolumn{1}{c|}{Min} &
          \multicolumn{1}{c|}{Ave} &
          \multicolumn{1}{c|}{Med} &
          \multicolumn{1}{c|}{Max} &
          \multicolumn{1}{c|}{Min} \\ \hline
    PSG & 13853 & 13867 & 14707 & 13007 & 10014 & 10014 & 10014 & 10014 & 3839 & 3853 & 4693 & 2993 \\ \hline

        \end{tabular}%
        }
        \caption{Token consumption details regarding Overall, Input, and Output in three stages. Batch ID: 4b06, Model: codestral:latest, Date: 05.21, Parameters: temperature=0.1, top\_p=0.3}
        \label{tab:token_details_4b06_codestral:latest_05.21}
        \end{table}
    
\newpage
\begin{figure}[!h]
    \centering
    \includegraphics[width=0.75\linewidth]{data_analysis/2025/05.21/tex/bar_plot_codestral:latest_05.21_4b06.pdf}
    \caption{Average Time and Token Consumption by Task Status. Model: codestral:latest, Date: 05.21, Parameters: temperature=0.1, top\_p=0.3 \\ \textit{The average time (top) and token consumption (bottom) are compared across three stages, broken down by overall performance (considering all runs), performance of only successful runs, and of only the failed runs.}}
    \label{fig:bar_plot_codestral:latest_05.21_4b06}
\end{figure}

\begin{figure}[!h]
    \centering
    \includegraphics[width=0.75\linewidth]{data_analysis/2025/05.21/tex/scatter_plot_codestral:latest_05.21_4b06.pdf}
    \caption{Scatter Plot Codestral:Latest 05.21 4B06Distribution of time and token consumption in three tasks. Model: codestral:latest, Date: 05.21, Parameters: temperature=0.1, top\_p=0.3}
    \label{fig:scatter_plot_codestral:latest_05.21_4b06}
\end{figure}


\clearpage 
 
\newpage
\subsection{Results: 07.28, Model: qwen2.5-coder:32b, Parameters: None}

\subsubsection{\texttt{batch\_uid:ae24}}

\FloatBarrier
\input{data_analysis/07.28/tex/tab_cost_details_qwen2.5-coder:32b_07.28_ae24.tex}

    \begin{table}[!h]
        \centering
        \resizebox{\textwidth}{!}{%
        \begin{tabular}{|l|r|r|r|r|r|r|r|}
        \hline
        \multicolumn{1}{|c|}{} &
          \multicolumn{3}{c|}{Ave Execution Time (s)} &
          \multicolumn{3}{c|}{Ave Total Tokens} &
          \multicolumn{1}{c|}{} \\ \cline{2-7}
        \multicolumn{1}{|c|}{\multirow{-2}{*}{Stages}} &
          \multicolumn{1}{c|}{Overall} &
          \multicolumn{1}{c|}{Pass} &
          \multicolumn{1}{c|}{Fail} &
          \multicolumn{1}{c|}{Overall} &
          \multicolumn{1}{c|}{Pass} &
          \multicolumn{1}{c|}{Fail} &
          \multicolumn{1}{|c|}{\multirow{-2}{*}{\shortstack{Success\\Rate}}} \\ \hline
    TPUSG & 69.69 & 61.30 & 187.17 & 5872 & 5207 & 15170 & 0.933 \\ \hline

        \end{tabular}%
        }
        \caption{Performance metrics of passed and failed runs,including execution time, token consumption, and success rate in three stages. Batch ID: ae24, Model: qwen2.5-coder:32b, Date: 07.28, Parameters: None}
        \label{tab:performance_metrics_ave_ae24_qwen2.5-coder:32b_07.28}
        \end{table}
    

    \begin{table}[!h]
        \centering
        \resizebox{\textwidth}{!}{%
        \begin{tabular}{|l|r|r|r|r|r|r|r|}
        \hline
        \multicolumn{1}{|c|}{} &
          \multicolumn{3}{c|}{Execution Time (s)} &
          \multicolumn{3}{c|}{Token Consumption} &
          \multicolumn{1}{c|}{} \\ \cline{2-7}
        \multicolumn{1}{|c|}{\multirow{-2}{*}{Stages}} &
          \multicolumn{1}{c|}{Ave} &
          \multicolumn{1}{c|}{Max} &
          \multicolumn{1}{c|}{Min} &
          \multicolumn{1}{c|}{Ave} &
          \multicolumn{1}{c|}{Max} &
          \multicolumn{1}{c|}{Min} &
          \multicolumn{1}{|c|}{\multirow{-2}{*}{\shortstack{Success\\Rate}}} \\ \hline
    TPUSG & 69.69 & 218.03 & 22.69 & 5872 & 15231 & 2587 & 0.933 \\ \hline

        \end{tabular}%
        }
        \caption{Performance metrics including execution time, token consumption, and success rate in three stages. Batch ID: ae24,Model: qwen2.5-coder:32b, Date: 07.28, Parameters: None}
        \label{tab:performance_metrics_ae24_qwen2.5-coder:32b_07.28}
        \end{table}
    

    \begin{table}[!h]
        \centering
        \begin{tabular}{|l|r|r|r|r|}
        \hline
        \multicolumn{1}{|c|}{Stages} &
          \multicolumn{1}{c|}{Ave} &
          \multicolumn{1}{c|}{Med} &
          \multicolumn{1}{c|}{Max} &
          \multicolumn{1}{c|}{Min} \\ \hline
    TPUSG & 69.69 & 47.10 & 218.03 & 22.69 \\ \hline

        \end{tabular}
        \caption{Time consumption (seconds) in three stages. Batch ID: ae24, Model: qwen2.5-coder:32b, Date: 07.28, Parameters: None}
        \label{tab:time_ae24_qwen2.5-coder:32b_07.28}
        \end{table}
    

    \begin{table}[!h]
        \centering
        \resizebox{\textwidth}{!}{%
        \begin{tabular}{|l|r|r|r|r|r|r|r|r|r|r|r|r|}
        \hline
        \multicolumn{1}{|c|}{} &
          \multicolumn{4}{c|}{Overall Token} &
          \multicolumn{4}{c|}{Input Token} &
          \multicolumn{4}{c|}{Output Token} \\ \cline{2-13}
        \multicolumn{1}{|c|}{\multirow{-2}{*}{Stages}} &
          \multicolumn{1}{c|}{Ave} &
          \multicolumn{1}{c|}{Med} &
          \multicolumn{1}{c|}{Max} &
          \multicolumn{1}{c|}{Min} &
          \multicolumn{1}{c|}{Ave} &
          \multicolumn{1}{c|}{Med} &
          \multicolumn{1}{c|}{Max} &
          \multicolumn{1}{c|}{Min} &
          \multicolumn{1}{c|}{Ave} &
          \multicolumn{1}{c|}{Med} &
          \multicolumn{1}{c|}{Max} &
          \multicolumn{1}{c|}{Min} \\ \hline
    TPUSG & 5872 & 2741 & 15231 & 2587 & 3935 & 1819 & 10011 & 1819 & 1936 & 922 & 5220 & 768 \\ \hline

        \end{tabular}%
        }
        \caption{Token consumption details regarding Overall, Input, and Output in three stages. Batch ID: ae24, Model: qwen2.5-coder:32b, Date: 07.28, Parameters: None}
        \label{tab:token_details_ae24_qwen2.5-coder:32b_07.28}
        \end{table}
    
\newpage
\begin{figure}[!h]
    \centering
    \includegraphics[width=0.75\linewidth]{data_analysis/07.28/tex/bar_plot_qwen2.5-coder:32b_07.28_ae24.pdf}
    \caption{Average Time and Token Consumption by Task Status. Model: qwen2.5-coder:32b, Date: 07.28, Parameters: None \\ \textit{The average time (top) and token consumption (bottom) are compared across three stages, broken down by overall performance (considering all runs), performance of only successful runs, and of only the failed runs.}}
    \label{fig:bar_plot_qwen2.5-coder:32b_07.28_ae24}
\end{figure}

\begin{figure}[!h]
    \centering
    \includegraphics[width=0.75\linewidth]{data_analysis/07.28/tex/scatter_plot_qwen2.5-coder:32b_07.28_ae24.pdf}
    \caption{Scatter Plot Qwen2.5-Coder:32B 07.28 Ae24Distribution of time and token consumption in three tasks. Model: qwen2.5-coder:32b, Date: 07.28, Parameters: None}
    \label{fig:scatter_plot_qwen2.5-coder:32b_07.28_ae24}
\end{figure}

\newpage
\subsection{Results: 07.28, Model: codestral:latest, Parameters: None}

\subsubsection{\texttt{batch\_uid:c8f6}}

\FloatBarrier

    \begin{table}[!h]
        \centering
        \resizebox{\textwidth}{!}{%
        \begin{tabular}{|l|r|r|r|r|r|r|r|r|r|r|r|r|}
        \hline
        \multicolumn{1}{|c|}{} &
          \multicolumn{4}{c|}{Overall Cost (USD cents)} &
          \multicolumn{4}{c|}{Input Cost (USD cents)} &
          \multicolumn{4}{c|}{Output Cost (USD cents)} \\ \cline{2-13}
        \multicolumn{1}{|c|}{\multirow{-2}{*}{Stages}} &
          \multicolumn{1}{c|}{Ave} &
          \multicolumn{1}{c|}{Med} &
          \multicolumn{1}{c|}{Max} &
          \multicolumn{1}{c|}{Min} &
          \multicolumn{1}{c|}{Ave} &
          \multicolumn{1}{c|}{Med} &
          \multicolumn{1}{c|}{Max} &
          \multicolumn{1}{c|}{Min} &
          \multicolumn{1}{c|}{Ave} &
          \multicolumn{1}{c|}{Med} &
          \multicolumn{1}{c|}{Max} &
          \multicolumn{1}{c|}{Min} \\ \hline
    TPUSG & 0.00 & 0.00 & 0.00 & 0.00 & 0.00 & 0.00 & 0.00 & 0.00 & 0.00 & 0.00 & 0.00 & 0.00 \\ \hline

        \end{tabular}%
        }
        \caption{Cost details regarding Overall, Input, and Output in three stages.  Batch ID: c8f6,Model: codestral:latest, Date: 07.28, Parameters: None}
        \label{tab:cost_details_c8f6_codestral:latest_07.28}
        \end{table}
    

    \begin{table}[!h]
        \centering
        \resizebox{\textwidth}{!}{%
        \begin{tabular}{|l|r|r|r|r|r|r|r|}
        \hline
        \multicolumn{1}{|c|}{} &
          \multicolumn{3}{c|}{Ave Execution Time (s)} &
          \multicolumn{3}{c|}{Ave Total Tokens} &
          \multicolumn{1}{c|}{} \\ \cline{2-7}
        \multicolumn{1}{|c|}{\multirow{-2}{*}{Stages}} &
          \multicolumn{1}{c|}{Overall} &
          \multicolumn{1}{c|}{Pass} &
          \multicolumn{1}{c|}{Fail} &
          \multicolumn{1}{c|}{Overall} &
          \multicolumn{1}{c|}{Pass} &
          \multicolumn{1}{c|}{Fail} &
          \multicolumn{1}{|c|}{\multirow{-2}{*}{\shortstack{Success\\Rate}}} \\ \hline
    TPUSG & 106.14 & 93.83 & 112.30 & 12911 & 10471 & 14132 & 0.333 \\ \hline

        \end{tabular}%
        }
        \caption{Performance metrics of passed and failed runs,including execution time, token consumption, and success rate in three stages. Batch ID: c8f6,Model: codestral:latest, Date: 07.28, Parameters: None}
        \label{tab:performance_metrics_ave_c8f6_codestral:latest_07.28}
        \end{table}
    

    \begin{table}[!h]
        \centering
        \resizebox{\textwidth}{!}{%
        \begin{tabular}{|l|r|r|r|r|r|r|r|}
        \hline
        \multicolumn{1}{|c|}{} &
          \multicolumn{3}{c|}{Execution Time (s)} &
          \multicolumn{3}{c|}{Token Consumption} &
          \multicolumn{1}{c|}{} \\ \cline{2-7}
        \multicolumn{1}{|c|}{\multirow{-2}{*}{Stages}} &
          \multicolumn{1}{c|}{Ave} &
          \multicolumn{1}{c|}{Max} &
          \multicolumn{1}{c|}{Min} &
          \multicolumn{1}{c|}{Ave} &
          \multicolumn{1}{c|}{Max} &
          \multicolumn{1}{c|}{Min} &
          \multicolumn{1}{|c|}{\multirow{-2}{*}{\shortstack{Success\\Rate}}} \\ \hline
    TPUSG & 106.14 & 149.09 & 25.13 & 12911 & 15565 & 2927 & 0.333 \\ \hline

        \end{tabular}%
        }
        \caption{Performance metrics including execution time, token consumption, and success rate in three stages. Batch ID: c8f6, Model: codestral:latest, Date: 07.28, Parameters: None}
        \label{tab:performance_metrics_c8f6_codestral:latest_07.28}
        \end{table}
    

    \begin{table}[!h]
        \centering
        \begin{tabular}{|l|r|r|r|r|}
        \hline
        \multicolumn{1}{|c|}{Stages} &
          \multicolumn{1}{c|}{Ave} &
          \multicolumn{1}{c|}{Med} &
          \multicolumn{1}{c|}{Max} &
          \multicolumn{1}{c|}{Min} \\ \hline
    TPUSG & 106.14 & 111.30 & 149.09 & 25.13 \\ \hline

        \end{tabular}
        \caption{Time consumption (seconds) in three stages. Batch ID: c8f6,Model: codestral:latest, Date: 07.28, Parameters: None}
        \label{tab:time_c8f6_codestral:latest_07.28}
        \end{table}
    

    \begin{table}[!h]
        \centering
        \resizebox{\textwidth}{!}{%
        \begin{tabular}{|l|r|r|r|r|r|r|r|r|r|r|r|r|}
        \hline
        \multicolumn{1}{|c|}{} &
          \multicolumn{4}{c|}{Overall Token} &
          \multicolumn{4}{c|}{Input Token} &
          \multicolumn{4}{c|}{Output Token} \\ \cline{2-13}
        \multicolumn{1}{|c|}{\multirow{-2}{*}{Stages}} &
          \multicolumn{1}{c|}{Ave} &
          \multicolumn{1}{c|}{Med} &
          \multicolumn{1}{c|}{Max} &
          \multicolumn{1}{c|}{Min} &
          \multicolumn{1}{c|}{Ave} &
          \multicolumn{1}{c|}{Med} &
          \multicolumn{1}{c|}{Max} &
          \multicolumn{1}{c|}{Min} &
          \multicolumn{1}{c|}{Ave} &
          \multicolumn{1}{c|}{Med} &
          \multicolumn{1}{c|}{Max} &
          \multicolumn{1}{c|}{Min} \\ \hline
    TPUSG & 12911 & 13829 & 15565 & 2927 & 9216 & 10240 & 10240 & 2048 & 3695 & 3856 & 5325 & 879 \\ \hline

        \end{tabular}%
        }
        \caption{Token consumption details regarding Overall, Input, and Output in three stages. Batch ID: c8f6, Model: codestral:latest, Date: 07.28, Parameters: None}
        \label{tab:token_details_c8f6_codestral:latest_07.28}
        \end{table}
    
\newpage
\begin{figure}[!h]
    \centering
    \includegraphics[width=0.75\linewidth]{data_analysis/07.28/tex/bar_plot_codestral:latest_07.28_c8f6.pdf}
    \caption{Average Time and Token Consumption by Task Status. Model: codestral:latest, Date: 07.28, Parameters: None \\ \textit{The average time (top) and token consumption (bottom) are compared across three stages, broken down by overall performance (considering all runs), performance of only successful runs, and of only the failed runs.}}
    \label{fig:bar_plot_codestral:latest_07.28_c8f6}
\end{figure}

\begin{figure}[!h]
    \centering
    \includegraphics[width=0.75\linewidth]{data_analysis/07.28/tex/scatter_plot_codestral:latest_07.28_c8f6.pdf}
    \caption{Scatter Plot Codestral:Latest 07.28 C8F6Distribution of time and token consumption in three tasks. Model: codestral:latest, Date: 07.28, Parameters: None}
    \label{fig:scatter_plot_codestral:latest_07.28_c8f6}
\end{figure}


\clearpage 

\newpage
\subsection{07.29, qwen2.5-coder:32b, Params: temperature=0.1, top-p=0.3}

\subsubsection{{PSG, TPUSG}; \texttt{batch\_uid:1974}}

\FloatBarrier
\input{data_analysis/2025/07.29/tex/tab_cost_details_qwen2.5-coder:32b_07.29_1974.tex}
\input{data_analysis/2025/07.29/tex/tab_performance_metrics_ave_qwen2.5-coder:32b_07.29_1974.tex}
\input{data_analysis/2025/07.29/tex/tab_performance_metrics_qwen2.5-coder:32b_07.29_1974.tex}
\input{data_analysis/2025/07.29/tex/tab_time_qwen2.5-coder:32b_07.29_1974.tex}
\input{data_analysis/2025/07.29/tex/tab_token_details_qwen2.5-coder:32b_07.29_1974.tex}
\newpage
\begin{figure}[!h]
    \centering
    \includegraphics[width=0.75\linewidth]{data_analysis/2025/07.29/tex/bar_plot_qwen2.5-coder:32b_07.29_1974.pdf}
    \caption{Average Time and Token Consumption by Task Status. Model: qwen2.5-coder:32b, Date: 07.29, Parameters: temperature=0.1, top-p=0.3 \\ \textit{The average time (top) and token consumption (bottom) are compared across three stages, broken down by overall performance (considering all runs), performance of only successful runs, and of only the failed runs.}}
    \label{fig:bar_plot_qwen2.5-coder:32b_07.29_1974}
\end{figure}

\begin{figure}[!h]
    \centering
    \includegraphics[width=0.75\linewidth]{data_analysis/2025/07.29/tex/scatter_plot_qwen2.5-coder:32b_07.29_1974.pdf}
    \caption{Distribution of time and token consumption in three tasks. Model: qwen2.5-coder:32b, Date: 07.29, Parameters: temperature=0.1, top-p=0.3}
    \label{fig:scatter_plot_qwen2.5-coder:32b_07.29_1974}
\end{figure}


\clearpage 


\input{data_analysis/07.30/tex/07.30}
\clearpage 


\input{data_analysis/07.30_a/tex/07.30_a}
\clearpage 

\newpage
\subsection{Results: 07.30, Model: gemma3:27b, Parameters: None}

\subsubsection{{PSG, TPUSG}; \texttt{batch\_uid:85a9}}

\FloatBarrier
\input{/data_analysis/2025/07.30_b/tex/tab_cost_details_gemma3:27b_07.30_b_85a9.tex}

    \begin{table}[!h]
        \centering
        \resizebox{\textwidth}{!}{%
        \begin{tabular}{|l|r|r|r|r|r|r|r|}
        \hline
        \multicolumn{1}{|c|}{} &
          \multicolumn{3}{c|}{Ave Execution Time (s)} &
          \multicolumn{3}{c|}{Ave Total Tokens} &
          \multicolumn{1}{c|}{} \\ \cline{2-7}
        \multicolumn{1}{|c|}{\multirow{-2}{*}{Stages}} &
          \multicolumn{1}{c|}{Overall} &
          \multicolumn{1}{c|}{Pass} &
          \multicolumn{1}{c|}{Fail} &
          \multicolumn{1}{c|}{Overall} &
          \multicolumn{1}{c|}{Pass} &
          \multicolumn{1}{c|}{Fail} &
          \multicolumn{1}{|c|}{\multirow{-2}{*}{\shortstack{Success\\Rate}}} \\ \hline
    PSG & 81.82 & 66.66 & 142.47 & 7677 & 6219 & 13507 & 0.800 \\ \hline
TPUSG & 96.12 & 42.33 & 97.97 & 7292 & 3070 & 7438 & 0.033 \\ \hline

        \end{tabular}%
        }
        \caption{Performance metrics of passed and failed runs,including execution time, token consumption, and success rate in three stages. Batch ID: 85a9, Model: gemma3:27b, Date: 07.30, Parameters: None}
        \label{tab:performance_metrics_ave_85a9_gemma3:27b_07.30}
        \end{table}
    

    \begin{table}[!h]
        \centering
        \resizebox{\textwidth}{!}{%
        \begin{tabular}{|l|r|r|r|r|r|r|r|}
        \hline
        \multicolumn{1}{|c|}{} &
          \multicolumn{3}{c|}{Execution Time (s)} &
          \multicolumn{3}{c|}{Token Consumption} &
          \multicolumn{1}{c|}{} \\ \cline{2-7}
        \multicolumn{1}{|c|}{\multirow{-2}{*}{Stages}} &
          \multicolumn{1}{c|}{Ave} &
          \multicolumn{1}{c|}{Max} &
          \multicolumn{1}{c|}{Min} &
          \multicolumn{1}{c|}{Ave} &
          \multicolumn{1}{c|}{Max} &
          \multicolumn{1}{c|}{Min} &
          \multicolumn{1}{|c|}{\multirow{-2}{*}{\shortstack{Success\\Rate}}} \\ \hline
    PSG & 81.82 & 155.17 & 26.90 & 7677 & 13863 & 2402 & 0.800 \\ \hline
TPUSG & 96.12 & 244.30 & 0.00 & 7292 & 15842 & 0 & 0.033 \\ \hline

        \end{tabular}%
        }
        \caption{Performance metrics including execution time, token consumption, and success rate in three stages. Batch ID: 85a9, Model: gemma3:27b, Date: 07.30, Parameters: None}
        \label{tab:performance_metrics_85a9_gemma3:27b_07.30}
        \end{table}
    

    \begin{table}[!h]
        \centering
        \begin{tabular}{|l|r|r|r|r|}
        \hline
        \multicolumn{1}{|c|}{Stages} &
          \multicolumn{1}{c|}{Ave} &
          \multicolumn{1}{c|}{Med} &
          \multicolumn{1}{c|}{Max} &
          \multicolumn{1}{c|}{Min} \\ \hline
    PSG & 81.82 & 76.68 & 155.17 & 26.90 \\ \hline
TPUSG & 96.12 & 21.16 & 244.30 & 0.00 \\ \hline

        \end{tabular}
        \caption{Time consumption (seconds) in three stages. Batch ID: 85a9, Model: gemma3:27b, Date: 07.30, Parameters: None}
        \label{tab:time_85a9_gemma3:27b_07.30}
        \end{table}
    

    \begin{table}[!h]
        \centering
        \resizebox{\textwidth}{!}{%
        \begin{tabular}{|l|r|r|r|r|r|r|r|r|r|r|r|r|}
        \hline
        \multicolumn{1}{|c|}{} &
          \multicolumn{4}{c|}{Overall Token} &
          \multicolumn{4}{c|}{Input Token} &
          \multicolumn{4}{c|}{Output Token} \\ \cline{2-13}
        \multicolumn{1}{|c|}{\multirow{-2}{*}{Stages}} &
          \multicolumn{1}{c|}{Ave} &
          \multicolumn{1}{c|}{Med} &
          \multicolumn{1}{c|}{Max} &
          \multicolumn{1}{c|}{Min} &
          \multicolumn{1}{c|}{Ave} &
          \multicolumn{1}{c|}{Med} &
          \multicolumn{1}{c|}{Max} &
          \multicolumn{1}{c|}{Min} &
          \multicolumn{1}{c|}{Ave} &
          \multicolumn{1}{c|}{Med} &
          \multicolumn{1}{c|}{Max} &
          \multicolumn{1}{c|}{Min} \\ \hline
    PSG & 7677 & 7738 & 13863 & 2402 & 5612 & 5817 & 9913 & 1721 & 2064 & 1921 & 3950 & 681 \\ \hline
TPUSG & 7292 & 1535 & 15842 & 0 & 4805 & 983 & 10158 & 0 & 2486 & 552 & 5684 & 0 \\ \hline

        \end{tabular}%
        }
        \caption{Token consumption details regarding Overall, Input, and Output in three stages. Batch ID: 85a9, Model: gemma3:27b, Date: 07.30, Parameters: None}
        \label{tab:token_details_85a9_gemma3:27b_07.30}
        \end{table}
    
\newpage
\begin{figure}[!h]
    \centering
    \includegraphics[width=0.75\linewidth]{/data_analysis/2025/07.30_b/tex/bar_plot_gemma3:27b_07.30_b_85a9.pdf}
    \caption{Average Time and Token Consumption by Task Status. Model: gemma3:27b, Date: 07.30, Parameters: None \\ \textit{The average time (top) and token consumption (bottom) are compared across three stages, broken down by overall performance (considering all runs), performance of only successful runs, and of only the failed runs.}}
    \label{fig:bar_plot_gemma3:27b_07.30_b_85a9}
\end{figure}

\begin{figure}[!h]
    \centering
    \includegraphics[width=0.75\linewidth]{/data_analysis/2025/07.30_b/tex/scatter_plot_gemma3:27b_07.30_b_85a9.pdf}
    \caption{Distribution of time and token consumption in three tasks. Model: gemma3:27b, Date: 07.30, Parameters: None}
    \label{fig:scatter_plot_gemma3:27b_07.30_b_85a9}
\end{figure}


\clearpage 

\newpage

\FloatBarrier
\section{Future Extensions}

\begin{itemize}
    \item Add library installation step to sketch generation. \textit{\\ Each code generation output can include a section, especially for imported libraries. The local executor installs those libraries for arduino-cli before trying compilation.}
    \item This is very interesting: \url{https://huggingface.co/blog/unified-tool-use}. I will write something about this in the current paper.

 \end{itemize}

\textcolor{gray}{\subsection{Extensions Mentioned in the Thesis}}


\begin{itemize}
    \item \textcolor{gray}{\textbf{Expanding Lifecycle Coverage}: Expanding the system to cover additional stages of the TinyML lifecycle, such as model designing and training, would provide a more comprehensive automation solution.}
    
    \item \textcolor{gray}{\textbf{Expanding Model and Hardware Coverage}: Testing this framework with a wider range of TinyML models and hardware platforms to assess its versatility and identify potential improvements.}
    
    \item \textcolor{gray}{\textbf{LLM Comparison}: Evaluating the performance of this framework with different LLMs to understand how the choice of LLM impacts the system's effectiveness.}
    
    \item \textcolor{gray}{\textbf{Performance Benchmarking}: Conducting comprehensive benchmarks comparing the LLM-powered approach to traditional development methods in terms of development time, code quality, and application performance would provide valuable insights into the system's practical benefits, and make the proposal more trustable.}
    
    \item \textcolor{gray}{\textbf{Qualitative Analysis}: Conducting a formal qualitative evaluation involving a questionnaire provided to different individuals for testing the proposed solution and comparing it with the traditional methods. This would provide valuable insights into real-world applicability.}
    
    \item \textcolor{gray}{\textbf{Improving Reliability}: Enhancing the success rate of code generation, particularly for sketch generation. This can be done by refining prompt engineering to have finer control of LLM's behavior.}
    
    \item \textcolor{gray}{\textbf{Specialized Fine-Tuning}: Fine-tuning LLMs specifically for TinyML tasks could improve LLM's performance and reliability in this domain.}
    
    \item \textcolor{gray}{\textbf{Integration with Traditional Tools}: Combining LLM-powered automation with traditional TinyML tools could leverage the strengths of both approaches.}
    
    \item \textcolor{gray}{\textbf{User Interface Development}: Creating intuitive interfaces for interacting with the LLM system could facilitate the usage of this system.}
\end{itemize}



\newpage
% \noindent\makebox[\linewidth]{\rule{\maxdimen}{0.4pt}}

\section{Bibliography List}
\cite{englhardtExploringCharacterizingLarge2023}
\cite{douWhatsWrongYour2024}
\cite{gaoEfficientToolUse2024}
% \maketitle

\bibliographystyle{ieeetr}
\bibliography{references}


\end{document}
